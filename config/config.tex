%**************************************************************
% file contenente dati automaticamente generati
%**************************************************************
\include{config/metadata.autogen}
%**************************************************************
% Frontespizio
%**************************************************************
\newcommand{\myName}{Matteo Di Pirro, Lisa Parma, Davide Polonio, Marco
Zanella} % autore
\newcommand{\myTitle}{Appunti di\\Gestione di Imprese Informatiche}
\newcommand{\myUni}{Università degli Studi di Padova}           % università
\newcommand{\myFaculty}{Corso di Laurea Magistrale in Informatica} % facoltà
\newcommand{\myDepartment}{Dipartimento di Matematica}          % dipartimento
\newcommand{\myLocation}{Padova}                                % dove
\newcommand{\myAA}{2017-2018}                                   % anno
\newcommand{\myCopyright}{CC-BY-SA-4.0}                         % copyright
\newcommand{\myRelease}{
https://github.com/Augugrumi/GestioneImpreseInformaticheNotes/releases}
\newcommand{\myIssue}{
https://github.com/Augugrumi/GestioneImpreseInformaticheNotes/issues}
\newcommand{\myPullRequest}{
https://github.com/Augugrumi/GestioneImpreseInformaticheNotes/pulls}

%**************************************************************
% Short cuts
%**************************************************************
\newcommand{\newglsacr}[2]{\newacronym[see={[Glossary:]{#1}}]{a-#1}{#1}{#2}}

%**************************************************************
% Impostazioni di impaginazione
% see: http://wwwcdf.pd.infn.it/AppuntiLinux/a2547.htm
%**************************************************************

\setlength{\parindent}{14pt}   % larghezza rientro della prima riga
\setlength{\parskip}{0pt}   % distanza tra i paragrafi


%**************************************************************
% Impostazioni di caption
%**************************************************************
\captionsetup{
    tableposition=top,
    figureposition=bottom,
    font=small,
    format=hang,
    labelfont=bf
}

%**************************************************************
% Impostazioni di glossaries
%**************************************************************
\makeglossaries
\input{res/glossary} % database di termini

%**************************************************************
% Impostazioni di graphicx
%**************************************************************
\graphicspath{{res/img/}} % cartella dove sono riposte le immagini


%**************************************************************
% Impostazioni di hyperref
%**************************************************************
\hypersetup{
    %hyperfootnotes=false,
    %pdfpagelabels,
    %draft,	% = elimina tutti i link (utile per stampe in bianco e nero)
    colorlinks=true,
    linktocpage=true,
    pdfstartpage=1,
    pdfstartview=FitV,
    % decommenta la riga seguente per avere link in nero (per esempio per la
%stampa in bianco e nero)
    %colorlinks=false, linktocpage=false, pdfborder={0 0 0}, pdfstartpage=1,
%pdfstartview=FitV,
    breaklinks=true,
    pdfpagemode=UseNone,
    pageanchor=true,
    pdfpagemode=UseOutlines,
    plainpages=false,
    bookmarksnumbered,
    bookmarksopen=true,
    bookmarksopenlevel=1,
    hypertexnames=true,
    pdfhighlight=/O,
    %nesting=true,
    %frenchlinks,
    urlcolor=webbrown,
    linkcolor=webbrown,
    citecolor=webgreen,
    %pagecolor=RoyalBlue,
    %urlcolor=Black, linkcolor=Black, citecolor=Black, %pagecolor=Black,
    pdftitle={\myTitle},
    pdfauthor={\textcopyright\ \myName, \myUni, \myFaculty},
    pdfsubject={},
    pdfkeywords={},
    pdfcreator={pdfLaTeX},
    pdfproducer={LaTeX}
}

%**************************************************************
% Impostazioni matematiche (esempi)
%**************************************************************
\newtheorem{exmp}{Esempio}[section]

%**************************************************************
% Impostazioni di itemize
%**************************************************************
%\renewcommand{\labelitemi}{$\ast$}

%\renewcommand{\labelitemi}{$\bullet$}
%\renewcommand{\labelitemii}{$\cdot$}
%\renewcommand{\labelitemiii}{$\diamond$}
%\renewcommand{\labelitemiv}{$\ast$}


%**************************************************************
% Impostazioni di listings
%**************************************************************
\lstset{
    language=[LaTeX]Tex,%C++,
    keywordstyle=\color{RoyalBlue}, %\bfseries,
    basicstyle=\small\ttfamily,
    %identifierstyle=\color{NavyBlue},
    commentstyle=\color{Green}\ttfamily,
    stringstyle=\rmfamily,
    numbers=none, %left,%
    numberstyle=\scriptsize, %\tiny
    stepnumber=5,
    numbersep=8pt,
    showstringspaces=false,
    breaklines=true,
    frameround=ftff,
    frame=single
}


%**************************************************************
% Impostazioni di xcolor
%**************************************************************
\definecolor{webgreen}{rgb}{0,.5,0}
\definecolor{webbrown}{rgb}{.6,0,0}
\definecolor{Pantone}{RGB}{155,0,20}
\definecolor{GrigioLight}{RGB}{152, 152, 152}


%**************************************************************
% Altro
%**************************************************************

\newcommand{\omissis}{[\dots\negthinspace]} % produce [...]

% eccezioni all'algoritmo di sillabazione
\hyphenation
{
    ma-cro-istru-zio-ne
    gi-ral-din
}

\newcommand{\sectionname}{sezione}
%\addto\captionsitalian{\renewcommand{\figurename}{figura}
%                       \renewcommand{\tablename}{tabella}}

\newcommand{\glsfirstoccur}{\ap{{[g]}}}

\newcommand{\intro}[1]{\emph{\textsf{#1}}}

%-------------------INIZIO creazione subsubparagraph---------------------------
\makeatletter
\newcounter{subsubparagraph}[subparagraph]
\def\toclevel@subsubparagraph{6}
\renewcommand\thesubsubparagraph{%
  \thesubparagraph.\@arabic\c@subsubparagraph}
\newcommand\subsubparagraph{%
  \@startsection{subsubparagraph}    % counter
    {6}                              % level
    {\parindent}                     % indent
    {3.25ex \@plus 1ex \@minus .2ex} % beforeskip
    {-1em}                           % afterskip
    {\normalfont\normalsize\bfseries}}
\newcommand\l@subsubparagraph{\@dottedtocline{6}{13.5em}{5em}}
\newcommand{\subsubparagraphmark}[1]{}
\setcounter{tocdepth}{6}
\setcounter{secnumdepth}{6} % aggiunge contatore ai paragrafi
\makeatother
%-------------------FINE creazione subsubparagraph-----------------------------

%-------------------Capitoli personalizzati------------------------------------

\titleformat{\chapter}[display]
  {\normalsize \huge  \color{black}}%
  {\flushright\normalsize \color{Pantone}%
   \MakeUppercase{\chaptertitlename}\hspace{1ex}%
   {\fontsize{60}{60}\selectfont\thechapter}}%
  {10 pt}%
  {\bfseries\huge#1}%

%----------------FINE Capitoli personalizzati----------------------------------

%----------------INIZIO Parte personalizzata-----------------------------------

\renewcommand\thepart{\Alph{part}}

\newcommand\partnumfont{% font specification for the number
  \fontsize{304}{104}\color{white}\selectfont%
}

\newcommand\partnamefont{% font specification for the name "PART"
  \color{white}\huge\bfseries%
}

\titleformat{\part}[display]
   {\normalfont\huge\filleft}
   { }
   {20pt}
   {\begin{tikzpicture}[remember picture,overlay]
  \fill[GrigioLight]
    (current page.north west) rectangle ([yshift=-13cm]current page.north
east);
    \node[
      fill=Pantone,
      text width=2\paperwidth,
      rounded corners=6cm,
      text depth=12cm,
      anchor=center,
      inner sep=0pt] at ([yshift=21cm]current page.south west) (parttop)
    {\thepart};%
    \node[
      anchor=center,
      inner sep=0pt,
      outer sep=0pt] at ([xshift=16cm, yshift=6cm]parttop.south) (partnum)
    {\partnumfont\thepart};%
    \node[
      anchor=north east,
      align=right,
      inner sep=0pt] at ([yshift=10cm] current page.center)
    {\parbox{.7\textwidth}{\raggedleft\partnamefont\MakeUppercase{#1}}};
    \end{tikzpicture}%
}

%--------------- FINE parte personalizzata ------------------------------------

%--------------- MULTI columns ------------------------------------------------

\newcounter{countitems}
\newcounter{nextitemizecount}
\newcommand{\setupcountitems}{%
  \stepcounter{nextitemizecount}%
  \setcounter{countitems}{0}%
  \preto\item{\stepcounter{countitems}}%
}
\makeatletter
\newcommand{\computecountitems}{%
  \edef\@currentlabel{\number\c@countitems}%
  \label{countitems@\number\numexpr\value{nextitemizecount}-1\relax}%
}
\newcommand{\nextitemizecount}{%
  \getrefnumber{countitems@\number\c@nextitemizecount}%
}
\newcommand{\previtemizecount}{%
  \getrefnumber{countitems@\number\numexpr\value{nextitemizecount}-1\relax}%
}
\makeatother
\newenvironment{AutoMultiColItemize}{%
\ifnumcomp{\nextitemizecount}{>}{3}{\begin{multicols}{2}}{}%
\setupcountitems\begin{itemize}}%
{\end{itemize}%
\unskip\computecountitems\ifnumcomp{\previtemizecount}{>}{3}{\end{multicols}}{}}

\newenvironment{AutoMultiColEnumerate}{%
\ifnumcomp{\nextitemizecount}{>}{3}{\begin{multicols}{2}}{}%
\setupcountitems\begin{enumerate}}%
{\end{enumerate}%
\unskip\computecountitems\ifnumcomp{\previtemizecount}{>}{3}{\end{multicols}}{}}

%--------------- END MULTI columns ---------------------------------------------

%---------------- DEFINITIONS ----------
\newtheoremstyle{definition}% name of the style to be used
  {\topsep}% measure of space to leave above the theorem. E.g.: 3pt
  {\topsep}% measure of space to leave below the theorem. E.g.: 3pt
  {\itshape}% name of font to use in the body of the theorem
  {0pt}% measure of space to indent
  {\bfseries}% name of head font
  {. ---}% punctuation between head and body
  { }% space after theorem head; " " = normal interword space
  {\thmname{#1}\thmnumber{ #2}\thmnote{ (#3)}}

\newtheorem{definition}{Definizione}[section]
