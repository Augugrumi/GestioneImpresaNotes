\chapter{Curriculum e assunzioni}

\section{La richiesta di lavoro}
Le aziende assumono per vari motivi e per prima cosa come esigenza: emerge
ovvero la necessità di una nuova risorsa umana. Quando ci si propone ad
un'azienda che non è alla ricerca di nuove figure spesso si perde
solamente tempo. La nature dell'esigenza può essere generate da due
motivazioni: la prima è che si è aperta una nuova posizione poiché l'azienda
vuole aumentare il personale in settori a lei noti, e in questo caso sarà
quindi in grado di supportare anche nuovi arrivati da un punto di vista tecnico,
oppure sta aprendo una nuova posizione in settori nuovi, in questo
caso è importante prestare attenzione in quanto probabilmente non si potrà far
affidamento su altre persone già esperte nel settore. In quest'ultima situazione
è più facile diventare una figura di riferimento all'interno dell'azienda, ma
essere l'unica persona a gestire un determinato ambito significa avere anche un
alto grado di responsabilità. Uno scenario completamente diverso è la ricerca
di un individuo a causa di una sostituzione: l'azienda ha quindi già esperienza
in quel settore, ma bisogna chiedersi come mai c'è stata la separazione col
precedente lavoratore. Nel caso l'azienda abbia avuto delle esperienze negative
con il precedente dipendente si possono avere molte possibilità di proporsi in
maniera positiva, altrimenti si possono avere scenari pericolosi, nel caso in
cui si debba rispettare una certa aspettativa lasciata dal lavoratore
precedente.

La \textbf{job description}, ovvero la descrizione del tipo di lavoratore che
si sta cercando deve essere precisa e descrittiva. Quando si va a fare un
colloquio di lavoro con qualcuno bisogna avere le idee molto chiare
dell'azienda alla quale ci si propone e per la posizione per la quale ci si
candida. Ruolo, skill e caratteristiche personali devono essere presenti nella
descrizione della richiesta di lavoro.
\todo{Qui ha fatto l'esempio di quello
che è stato assunto per fare una cosa ma ne finisce per fare un'altra per
evidenziare come anche la definizione del posto sia importante}

\section{Il curriculum}
È il primo strumento di selezione: esiste di tipo cartaceo o digitale. Oramai
il curriculum è quasi sempre digitale, e si usa quello cartaceo durante il
colloquio meramente per prenderci appunti.

Il curriculum non può essere identico per tutte le posizioni lavorative. Una
buona pratica è enfatizzare le esperienze che si hanno negli ambiti specifici
in cui l'azienda sta cercando lavoro. L'attenzione di chi legge il curriculum
va catturata sfruttando parole evidenziate e spiegando maggiormente i punti più
importanti. Questo non significa mettere tutto nel curriculum, in quanto
potrebbe distrarre chi lo legge dai punti focali. Il livello di dettaglio
dovrebbe variare in base al tipo di curriculum, all'esperienza e alla posizione
per la quale ci si candida.

Il curriculum non è un mero diario di quello che si è fatto nella propria vita.
Ad esempio formazioni professionali distanti da quelle cercate nell'azienda non
sono utili, soprattutto nel caso in cui si abbia già maturato una certa
esperienza, ed è meglio porle in secondo piano, o potrebbero risultare
fuorvianti.

Non mentire nel proprio curriculum è fondamentale: piuttosto è
meglio scrivere sotto un'altra luce le esperienze fatte. Diversamente da altri
paesi, in Italia è un aspetto cruciale.

Riguardo la foto sul curriculum, ci può stare come no: in ogni caso, trasmette
tantissimo. L'estetica è uno dei primi elementi con cui le gente viene
valutata, a causa del mid brain che interviene nel nostro processo decisionale.
Nei ruoli dove il contatto con il pubblico è importante la foto deve essere
presente, ma non è necessaria in lavori di ``backend'', dove l'aspetto non è
necessario, metterla quindi non è obbligatorio ma potrebbe diventare un'arma
a doppio taglio.

\subsection{Social network}
Riguardo ai social network, bisogna prestare un po' di attenzione. LinkedIn va
tenuto sempre in ordine, mentre su Facebook bisogna controllare quali sono i
contenuti visibili a chiunque.

\subsection{Lettera di presentazione}
Una lettera di presentazione è una lettera che introduce una persona e ne
esalta le caratteristiche per quel determinato posto di lavoro. Rappresenta
tutte le caratteristiche emotive ed emozionali che non sono inseribili nel
curriculum, ed in certi casi è altrettanto importante del curriculum, e
può essere di accompagnamento. Anche questa, non dovrebbe essere né asettica né
copia-incollata. In una lettera di presentazione bisogna soprattutto mettere in
evidenza quanto quell'azienda è importante per il candidato, non il contrario.
