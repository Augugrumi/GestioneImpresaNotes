\newpage
\chapter{Business Model}

Una startup è un \textbf{progetto di business}, non un'iniziativa di
innovazione.
Un investitore infatti è interessato alla capacità di una startup di
fare business e questo è ciò che ne determina nel medio periodo la
sopravvivenza.

Fare business non significa ottenere i soldi da un investitore. Significa
ottenere un guadagno (sia per lo startupper che per l'investitore) a partire
dai soldi investiti.

Fare business non significa nemmeno avere una base di utenti cospicua: da
sola non basta per reggere in piedi una startup. Occorre generare un guadagno
da tutti gli utenti.

Un business model descrive il modo in cui un'impresa genera valore per i suoi
clienti e \textbf{trasforma questi valori in ricavi}. Aiuta anche a capire:

\begin{itemize}

\item cosa si sta costruendo?

\item cosa serve per costruirlo?

\item a chi si sta proponendo?

\item come trasformare i valori offerti in ricavo?

\end{itemize}

Un business model è il progetto di ciò che si vuole costruire e aiuta a
direzionare gli sforzi verso azioni in linea con gli obiettivi che si vogliono
raggiungere.

Una rappresentazione di business model è stata proposta da Alexander
Osterwalder e Yves Pigneur: una cornice chiamata \textbf{business canvas} viene
suddivisa in 9 blocchi.

\subsection{Value proposition - VP}

Si tratta del blocco centrale del modello e divide gli aspetti produttivi (sx)
da quelli commerciali (dx).

Ciò che una startup offre al mercato non è semplicemente il prodotto/servizio
che realizza ma \textbf{i benefici generati per i clienti}.

Ad esempio, chi compra una Ferrari non compra solo un'auto di lusso, ma entra a
far parte di un'élite e ottiene uno status symbol.

Altro esempio: chi acquista una Toyota Prius (una delle auto ibride più
avanzate sul mercato) può farlo perché è attento al consumo oppure perché vuole
far capire che è attento all'ambiente.

La VP consiste nell'insieme di bisogni che i prodotti delle startup soddisfano
e nel modo in cui li soddisfano.

Può capitare che un cliente scelga il prodotto di una startup per ragioni
diverse dalle quali era stato inizialmente concepito. In questo caso occorre
capire quali sono i \textbf{reali bisogni} dei clienti.

Le VP possono essere ricondotte a 11 categorie generali.

\begin{itemize}

\item \textbf{Novità}: VP che soddisfa nuovi bisogni che in precedenza i
clienti non percepivano come necessari (esempio: pubblicare le proprie attività
quotidiane su Instagram).

\item \textbf{Performance}: VP che migliora prodotti vecchi (ad esempio di auto
o schede grafiche). Questa VP ha un limite che può essere legale o fisico.

\item \textbf{Personalizzazione}: VP che soddisfa specifiche esigenze personali
(esempio: le camicie su misura). Di recente è diventata una tendenza comune la
personalizzazione di massa, dove viene fornito un insieme limitato di opzioni
tra cui scegliere, creando prodotti \textbf{quasi} unici.

\item \textbf{Supporto}: VP che aiuta il cliente a ottenere un risultato
(esempio: consulenza).

\item \textbf{Design}: VP che punta a un design accattivante (ad esempio, nella
moda). Il concetto di bellezza cambia in base alla cultura e agli individui ed
evolve nel tempo.

\item \textbf{Brand e status} VP che mette in risalto il marchio e i suoi valori
(ad esempio Coca Cola).

\item \textbf{Prezzo} VP che punta sul prezzo (offrire lo stesso servizio a un
costo più basso rispetto alla concorrenza). Ikea e Ryanair hanno questa VP.

\item \textbf{Riduzione dei costi} VP che aiuta i clienti a ridurre i loro
costi operativi (ad esempio, i software gestionali).

\item \textbf{Riduzione dei rischi} VP che permette di tutelarsi a fronte di
possibili rischi (esempio: compagnie assicurative).

\item \textbf{Accessibilità} VP che punta a rendere disponibili dei prodotti
o servizi a fasce di clienti che prima non potevano accedervi (ad esempio
connettività radio vs connettività cablata).

\item \textbf{Usabilità} VP che punta a rendere dei prodotti più facili da
usare (ad esempio la Nintendo Wii ha delle performance hardware modeste ma è
molto intuitiva e così ha saputo attirare l'attenzione di utenti non abituati
all'uso di joystick e joypad, aprendo una fascia di mercato nuova - casual
gamers e famiglie).

\end{itemize}

\subsection{Customer segment - CS}

Definiscono i diversi gruppi di persone o organizzazioni che il business si
propone di raggiungere e servire.

Agli occhi degli investitori risulta più interessante e apprezzato un prodotto
che mira a una nicchia limitata, ma ben definita, di clienti, piuttosto che un
prodotto di massa.

Individuare con precisione un target permette di capire i suoi veri bisogni.

Bisogna studiare bene il settore, confrontarsi con i possibili clienti ed
essere sempre pronti, se necessario, ad abbandonare l'idea iniziale per
adattarla alle reali necessità del mercato.

\subsection{Canali - CH}

I canali sono i modi in cui la startup comunica e raggiunge il suo segmento
di mercato, veicolano tutte le informazioni e i contatti tra l'azienda e i
clienti (esempio: Internet, tv, giornali, punti vendita, fiere, ecc\dots).

Una volta chiarito il CS, occorre scegliere bene i canali. Quali canali sono
i migliori per i clienti che ho individuato? Che ambienti frequentano, come
si comportano, chi influenza le loro scelte?

Il processo di acquisto si acquista su 5 fasi distinte:

\begin{itemize}

\item \textbf{Consapevolezza}. Bisogna far sapere ai clienti che la
startup/azienda esiste. Di solito si usa la pubblicità tradizionale, tv,
giornali, Internet o fiere. Una delle difficoltà principali delle startup
è quella di essere completamente sconosciuta (o venire ignorata) appena
avviata.

\item \textbf{Valutazione}. I cliente, una volta che hanno conosciuto la
startup/azienda valutano la sua VP. In questa fase entrano in gioco i
cataloghi, i listini, le recensioni, il passaparola (esempio: Booking.com
che trova il suo punto di forza nelle recensioni lasciate dai clienti).

\item \textbf{Acquisto} L'acquisto può avvenire online, fisicamente, tramite
un contratto a domicilio o per posta. L'importante è che sia il meno complicato
possibile (esempio: Amazon 1-click).

\item \textbf{Consegna} Come verrà consegnato il prodotto/servizio? Quali sono
le tempistiche e i rischi? Occorre tenere informato il cliente durante questa
fase (a meno che l'acquisto non sia stato fatto in un negozio fisico e il
prodotto fosse subito disponibile).

\item \textbf{Post vendita} Finché un cliente utilizzerà un prodotto dovrà
essere tenuto aperto un canale di comunicazione con lui (ad esempio per
l'assistenza).

\end{itemize}

\subsection{Customer relationship - CR}

Descrive il tipo di rapporto che un'organizzazione ha con i suoi clienti
(passati, presenti e futuri).

Una relazione può essere\dots

\begin{itemize}

\item basata sull'assistenza personale (prima, durante o dopo il processo di
acquisto).

\item basata sull'assistenza personale dedicata (come la prima, ma in questo
caso ci sarà sempre la stessa persona per un cliente - come nel rapporto
medio/paziente). Questa relazione si basa sulla fiducia.

\item self-service. Questa è una non-relazione: si dà al cliente un libretto
delle istruzioni/cd e si spera che basti a risolvere eventuali dubbi/problemi
(esempio Ikea).

\item self-service + supporto automatico. Qui non c'è relazione con un essere
umano, ma con una macchina/software.

\item di comunità. Se attorno a un prodotto si è creata una community, spesso e
volentieri ci si rivolge ai suoi membri per consigli/suggerimenti (esempio:
gaming online)

\item Co-creazione: alcune organizzazioni si fanno aiutare dai clienti stessi
a individuare e creare valore (esempio: crowdsourcing).

\end{itemize}

La difficoltà è quella di individuare una relazione soddisfacente per il
cliente e che si integri al resto del business model.

\subsection{Key resource - KR}

Le key resource (attività chiave) sono gli asset necessari per far funzionare
un modello di business.
Cosa è necessario per creare il prodotto, farsi conoscere dai clienti e
consegnare loro la VP?
Esistono 4 tipologie di risorse:

\begin{itemize}
 \item \textbf{Risorse fisiche}
 \item \textbf{Risorse umane}
 \item \textbf{Risorse intellettuali}
 \item \textbf{Risorse finanziarie}
\end{itemize}

\subsection{Key activities - KA}

Le key activities trasformano le KR in VP. Sono le attività cruciali che
l'organizzazione deve seguire per far funzionare il modello di business.
Possono essere attività di tipo produttivo (riguardanti la progettazione e
realizzazione di un prodotto), problem solving (risolvere un problema specifico
sottoposta dal cliente), gestione e promozione. Ad esempio, per Facebook una
delle attività chiave è cambiare e ridisegnarsi per mantenere alto l'interesse
dei suoi utenti.

\subsection{Key partnership - KP}

Spesso le startup non generano tutto internamente, ma si inseriscono in filiere
produttive. Le partnership possono servire anche per espandere il proprio
business e raggiungere mercati altrimenti fuori portata.

Una startup non deve considerare il resto del mondo come un nemico: trovare
alleati è un aiuto importante.

Esistono 4 tipi di partnership:

\begin{itemize}
 \item \textbf{Cliente-Fornitore}: questa relazione si basa sulla
fedeltà dei clienti acquisiti.
 \item \textbf{Alleanza strategica tra non-competitor}: relazione
tra due aziende non competitor, che possono però mirare alla stessa
nicchia di clienti.
 \item \textbf{Cooperazione}: alleanza tra competitor che può avvenire
per vari motivi (suddivisione dei rischi, definire un nuovo standard,
attaccare un leader di mercato\dots)
 \item \textbf{Joint venture}: accordo puramente finanziario, ad esempio
tra una startup e Venture Capital o Business Angel.
\end{itemize}

\subsection{Cost structure - C\$}

La cost structure descrive tutti i costi sostenuti per operare in un modello di
business. Creare una VP, consegnarla ai clienti e mantenere con loro una
relazione ha un costo di cui deve essere tenuto conto quando si hanno gli
investimenti.

Esistono due tipi di business model che possono aiutare a capire come
suddividere le spese:

\begin{itemize}
 \item Modelli \textbf{cost driven}: business pensati per offrire ai clienti
la VP con il costo più basso possibile. Per questi modelli sono importanti
il mantenimento di strutture snelle, l'autoproduzione e l'outsourcing.
 \item Modelli \textbf{value driven}: business in cui la qualità del prodotto
è più importante che mantenere un prezzo basso (ad esempio, hotel a 5 stelle).
\end{itemize}

\subsection{Revenue stream - R\$}

Individuare il pricing corretto per un prodotto/servizio è un'attività
cruciale. Un approccio comune per business consolidati è partire dal costo di
produzione della VP e aggiungerci il margine che si vuole ottenere.
Per business che devono decollare si parte dai clienti: quanto sono disposti a
pagare per la VP offerta? A quale aspetto sono interessati? La risposta
a questa domanda farà conoscere i flussi di incassi per ogni segmento
(di clienti) che dovranno coprire i costi e generare margine per permettere
alla start up di guadagnare.

\subsection{Il percorso del valore}

Non sempre tutti i blocchi del business model sono critici, occorre avere
chiaro quali sono quelli primari e quali quelli secondari. Si deve capire come
sono collegati e qual à il percorso che genera valore. Una configurazione
comune parte dai costi passando per i blocchi produttivi che generano proposta
di valore, che viene consegnata ai clienti, che pagando, generano ricavi.

Attenzione, questo percorso non stabilisce in che ordine vanno compilati i
vari blocchi del Business Model, non ha nulla a che fare con il suo design!

\subsection{Business Model design}

Come ideare un Business Model? La prima cosa da fare è studiare molto bene il
settore in cui si vuole lavorare fino a conoscerlo alla perfezione.
Bisogna capire chi sono i players, grandi e piccoli. Studiando il loro
percorso nel settore e gli eventuali errori che hanno commesso si può
evitare di fare gli stessi sbagli e risparmiare del tempo. Bisogna anche
individuare quali sono i trend sociali e tecnologici del settore.
Sono tre gli aspetti su cui focalizzarsi in questa fase iniziale:

\begin{itemize}

\item \textbf{I competitor} Chi è stato/è presente nel mercato?
Cosa ha fatto di giusto? Che eventuali errori ha commesso?

\item \textbf{Le tecnologie} Quali tecnologie possono essere utili al progetto?
Qual è il loro costo?

\item \textbf{I trend} Cosa succederà tra qualche anno, quali saranno gli
sviluppi possibili? Verso dove stanno puntando gli aspetti legislativi e
normativi?

Una volta studiati questi elementi si potrà ideare qualcosa di nuovo,
generando molte idee e identificando le migliori.
Le idee vanno analizzate, valutando pregi e difetti, originalità e
fattibilità e la passione che esse suscitano.

Quando si hanno una manciata di idee potenzialmente interessanti vanno
costruiti i primi Business Model.

Per ogni blocco dei Business Model realizzati ci si deve chiedere quali
sono i suoi punti di forza e le debolezze, quali sono i requisiti
necessari, quali gli aspetti innovativi, cosa manca nel mercato che si
possiede?

La selezione finale per la scelta di un'unica idea va fatta in base
(di nuovo) a forze, debolezze, fattibilità tecnologica e sostenibilità
economica.

\end{itemize}
