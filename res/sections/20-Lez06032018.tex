\chapter{Start-up}

Che cos'è? È un'azienda giovane creata da profili giovani. Di conseguenza tende
ad avere una visione più ``giovane'', e 2-3 anni di esperienza possono fare di
un programmatore ``novello'' uno un po' più formato.
Prima di poter uscire dalla categoria ``junior'' comunque, sono richiesti da 5
ai 10 anni di esperienza, dove è importante saper ascoltare e capire le
dinamiche interne di un'azienda. È possibile notare come il percorso sia
quindi lungo.

\section{Introduzione}

\paragraph*{Definizione} S'intende l'operazione e il periodo (solitamente 2-3
anni) durante il quale si avvia un'impresa. Ovvero è un'azienda che sta vivendo
il suo periodo d'avvio d'impresa. Negli ultimi anni il senso di questo termine
si è ampliato: infatti si applica anche ad aziende più strutturate, anche per
indicare espansioni di aziende storiche (ad esempio da nuove invenzioni possono
derivare creazioni di \textit{spin-off aziendali}).

\subparagraph*{Spin-off} \todo{Il prof ha fatto dei giri col discorso e
siamo finiti a parlare di pubblicità, era sempre relativo alle spin-off}

Comunicazione \textbf{mirroring}: significa ``comunicare a qualcuno nello
stesso modo in cui lui comunicherebbe con te''.

Questa comunicazione risulta efficace in quanto una persona tende a fidarsi di
più di persone tendenzialmente ``simili'' a lui/lei.

Un'azienda che vuole lanciare un nuovo prodotto e vuole caratterizzarlo in
qualche maniera potrebbe a volte lanciarlo creando uno spin-off aziendale.
Questo potrebbe incrociare nuovi interessi da parte dei
venditori.\footnote{Questo accade anche a livello universitario per esempio, ma
bisogna stare attenti quando si è in ambito pubblico.}

\subparagraph*{Le vere e proprie start-up} Le vere e proprie start-up sono di
origine anglosassone. Come viene definito da Paul Graham:
\emph{
Una start-up è un'azienda pensata per crescere in fretta. Il fatto di essere
stata fondata di recente non rende di per sé un'azienda una start-up. Né è
necessario che la start-up lavori nella tecnologia o ottenga fondi da Venture
Capital o che abbia una qualche formati di ``exit''. L'unica cosa essenziale è
la crescita. Tutto il resto che noi associamo alle start-up segue la crescita.
}
È vero che un'azienda neo-nata tende a crescere più velocemente rispetto ad
un'azienda storica, soprattutto per chi opera nel settore tecnologico, perché
spesso propongono cose, concetti, strumenti che molto prima non esistevano.
Quindi queste start-up si propongono su mercati molto più ampi da conquistare.

\section{Creare una start-up di successo}

È possibile che un giorno la mia start-up diventi di successo e che ne possa
guadagnare qualcosa? Si, è possibile, ma è molto difficile.
Il problema del mondo latino/europeo/italiano è che sono presenti modelli
imprenditoriali radicalmente diversi dal mondo anglosassone.
Nel mondo italiano, il percorso tipico di un'azienda è:
\begin{itemize}
 \item Inizia dal fondatore che apre l'azienda
 \item Dopo una certa età, l'azienda passa in mano al figlio
 \item \dots
\end{itemize}
Quindi, generazione dopo generazione l'azienda passa di padre in figlio,
diversamente dal modello anglosassone dove dopo la fondazione di
un'azienda di solito il fondatore vende il suo progetto e ne inizia una nuova,
capitalizzandoci sopra. Qui il numero di imprenditori che un'azienda ha è molto
diverso.

Di questi due modelli non ce n'è uno migliore, ma si hanno caratteristiche
molto diverse: nel primo caso, l'azienda è una creatura dell'imprenditore,
considerata quasi come un ``figlio'', segnalando un legame molto profondo tra
lui e la sua realtà e un suo fallimento corrisponde a una sconfitta personale,
il secondo l'azienda è solo un progetto e il fallimento viene vista come
un'esperienza.
