\chapter{Start-up}

Che cos'è una \textit{startup}? Generalmente si intende un'azienda giovane
creata da profili giovani. Di conseguenza tende ad avere una visione più
``fresca'' di lavoro e azienda. In questo contesto in 2-3 anni di esperienza
un programmatore non è più \textit{junior}, ma è considerato già \textit{mid}
o addirittura \textit{senior}. In aziende più strutturate invece il tempo
per abbandonare la categoria \textit{junior} è molto più lungo e varia dai 5
ai 10 anni (in alcuni contesti addirittura 15). Soprattutto quando si è in
questa categoria è necessario saper ascoltare i programmatori più esperti e
capire le dinamiche interne di un'azienda, facendo attenzione nel caso in
cui si vogliano proporre nuove idee, ancora di più nel caso in cui non si è
stati assunti per portare qualcosa di ``nuovo''. Le idee proposte potrebbero
essere considerate infatti:
\begin{itemize}
  \item fuori contesto;
  \item ``vecchie'', perché già esplorate dall'azienda;
  \item che non hanno una visione di insieme del contesto aziendale anche se
  valide.
\end{itemize}
Ciò potrebbe essere visto quindi in malo modo all'interno dell'azienda dai
lavoratori più anziani.

\section{Introduzione}

\paragraph*{Startup come ``operazione''} Per \textit{startup} s'intende
l'operazione e il periodo (solitamente di 2-3 anni) durante il quale si avvia
un'impresa. Seguendo questa definizione, quindi, possono essere definite come
startup tutte le aziende che sono in fase di apertura (anche una pizzeria per
esempio).

\paragraph*{Startup come ``spin-off''}
%\todo{Il prof ha fatto dei giri col discorso e siamo finiti a parlare di
%pubblicità, era sempre relativo alle spin-off}
Negli ultimi anni il significato di \textit{startup} è stato ampliato: infatti
si applica anche ad aziende più strutturate che creano delle espansioni (ad
esempio a seguito di nuove invenzioni) chiamate \textit{spin-off}. Solitamente
vendono prodotti differenti dall'azienda originale e agiscono in mercati
diversi: ciò determina differenti pubblicità, comunicazione e soprattutto
una differente rete di vendita. Proprio la differenziazione sulla rete di
vendita determina l'impossibilità di mantenere un'unica azienda.

\begin{example}[Azienda di mattonelle]\label{exmpl:startup:mattonelle}
Abbiamo un'azienda di prodotti per l'edilizia che produce mattonelle per
lastricare il terreno. Solitamente in questo campo non ci sono grandi
differenze tra i vari competitor poiché ci sono delle norme da seguire quindi
la differenza sta nel prezzo. Per questo motivo la pubblicità è un blocco molto
grande in cui il prezzo è la cosa più importante e non viene considerato il
fattore estetico. Il reparto vendite si preoccuperà principalmente di parlare
con capocantieri. Se l'azienda successivamente inizia a produrre mattonelle
simil legno (un prodotto di lusso) allora dovrà cambiare il modo in cui
pubblicizza tale prodotto, per il quale dovrà essere esaltato il fattore
estetico, e cambierà anche a chi si rivolgerà il reparto vendite: in
questo caso a architetti e designer.
\end{example}

\noindent Una delle tecniche utili per le vendite è il \textbf{mirroring}.
\begin{definition}[Mirroring]
 Per mirroring s'intende ``comunicare a qualcuno nello stesso modo in cui lui
comunicherebbe con te''
\end{definition}

\noindent La comunicazione tramite mirroring risulta più efficace in quanto una
persona tende a fidarsi di più di altre tendenzialmente ``simili'' a lui/lei.
Questo però è impossibile se ho un'azienda che produce sia beni di lusso che
semplici mattonelle come nell'Esempio~\ref{exmpl:startup:mattonelle}, creerebbe
danno d'immagine: il capocantiere vedrebbe l'azienda come quella che produce
cose costose, l'architetto come quella che produce semplici mattonelle.

Altro tipo di spin-off sono quelle universitarie: tali aziende rischiano spesso
di fare più male che bene, poiché tolgono fondi all'università e generano una
sorta di concorrenza sleale. Infatti tali aziende spesso non riuscirebbero a
resistere senza i fondi spesi dall'università.

\paragraph*{Le vere e proprie start-up} Le vere e proprie start-up sono di
origine anglosassone e nascono negli anni '70. Paul Graham le definisce:
\emph{
Una start-up è un'azienda pensata per crescere in fretta. Il fatto di essere
stata fondata di recente non rende di per sé un'azienda una start-up. Né è
necessario che la start-up lavori nella tecnologia o ottenga fondi da Venture
Capital o che abbia una qualche forma di ``exit''. L'unica cosa essenziale è
la crescita. Tutto il resto che noi associamo alle start-up segue la crescita.
}
Quindi la cosa più importante per una start-up è la crescita veloce.
È vero che un'azienda neo-nata tende a crescere più velocemente rispetto ad
un'azienda storica, soprattutto per chi opera nel settore tecnologico, perché
spesso propongono cose, concetti, strumenti che molto prima non esistevano.
Quindi queste start-up si propongono su mercati molto più ampi da conquistare
proponendo spesso qualcosa che prima non c'era.

\section{I modelli imprenditoriali italiano e anglosassone}
È possibile che un giorno la mia start-up diventi di successo e che ne possa
guadagnare qualcosa? Si, è possibile, ma è molto difficile.
Il problema del mondo latino/europeo/italiano è che sono presenti modelli
imprenditoriali radicalmente diversi dal mondo anglosassone e difatti
le startup sono una realtà relativamente nuova in Italia (da 5/10 anni).
Nel mondo italiano, il percorso tipico di un'azienda è:
\begin{enumerate}
 \item Un fondatore apre l'azienda;
 \item Il fondatore la dirige per un cospicuo numero di anni;
 \item Dopo una certa età, l'azienda passa in mano al figlio;
 \item \dots
\end{enumerate}
Quindi, generazione dopo generazione l'azienda passa di padre in figlio.
Il modello anglosassone è invece differente:
\begin{enumerate}
 \item Un fondatore apre l'azienda;
 \item Il fondatore la dirige per una decina d'anni;
 \item L'azienda viene venduta e ne viene fondata una nuova;
 \item \dots
\end{enumerate}
Di questi due modelli non ce n'è uno migliore, ma si hanno caratteristiche
molto diverse: nel modello italiano si ha una stessa azienda che viene passata
a più imprenditori, di padre in figlio, ed è considerata dall'imprenditore come
un ``figlio'', e c'è un legame molto profondo tra imprenditore, azienda e
magari anche lavoratori. In questo caso un fallimento corrisponde ad una
sconfitta personale, un disonore per la famiglia, un problema lasciare senza
lavoro dei dipendenti a cui l'imprenditore è affezionato. Nel modello
anglosassone, invece, si ha uno stesso imprenditore che dà vita a molteplici
aziende. Una società è un progetto che l'imprenditore porta dove vuole e poi
vende. Il fallimento, in questo contesto, non è un problema ma viene visto come
dell'esperienza fatta dall'imprenditore\footnote{In media infatti il successo
di un imprenditore si ha dopo il terzo fallimento}, e, al contrario, i
\textit{venture capitalist} sono propensi a sostenere qualcuno che ha già avuto
un progetto che è fallito in quanto, sperabilmente, gli errori che hanno
portato al fallimento non verranno più commessi. In Italia anche questo è
completamente differente, in quanto nel caso di fallimento una banca non
sosterrà più altri progetti dello stesso individuo.
