Proseguiamo ora a vedere gli ultimi punti della value proposition.

\paragraph*{Riduzione dei rischi} Per un cliente adottare un nuovo prodotto
implica un rischio (cedimenti tecnologici, malfunzionamento o servizi di
qualità ignota). Ridurre il rischio dovuto alla nuova adozione è importante.
Infatti, ogni qualvolta si cambia un prodotto il cliente deve sostenere il
\emph{costo di commutazione}: questo per esempio prevede l'imparare ad usare il
nuovo prodotto, sostituire applicazione presenti con il precedente prodotto e
non presenti nel nuovo (quando per esempio si passa da Apple ad Android) e così
via. Tale costo è ancora più importante se si è un'azienda, poiché va
moltiplicato per il numero di dipendenti. In questo tipo di value proposition
possono risultare importanti:
\begin{itemize}
\item garanzie;
\item supporto o/e servizio clienti;
\item reputazione del marchio;
\item assicurazioni.
\end{itemize}

\paragraph*{Accessibilità} Rendere prodotti o servizi disponibili a fasce
di clienti che precedentemente non potevano accedervi. Un semplice esempio
potrebbe essere l'apertura di un ristorante etnico dove non ce ne sono di
quella tipologia. Questa value proposition può essere molto debole, in quanto
è facile che nascano concorrenti, annullando il vantaggio iniziale. In tal
caso, quindi, è necessario passare ad altri tipi di value proposition.
Anche l'apertura di centri commerciali e il posizionamento di negozi/servizi in
aree strategiche si può basare sull'accessibilità.

\paragraph*{Usabilità} Rendere i prodotti più facili da usare può aprire a
fasce di clienti completamente nuove. Come esempi troviamo i primi iPhone e,
soprattutto, la Nintendo Wii. Quest'ultima ha aperto un mercato totalmente
nuovo, il \emph{family gaming}, semplificando il joystick, creando giochi
che non richiedono un hardware importante e abbassando il costo della
console.

\todo{Il prof aveva parlato di USP e ha detto che la si trova nel libro.
Sarebbe figo avere un riassuntino qui.}

\chapter{Public Speaking}

Saper parlare in pubblico è importante, perché permette di vendere meglio il
proprio prodotto ma anche per capire quando qualcuno sta cercando di vendere
``fumo'' con belle parole. Probabilmente le prime esperienze non sono positive,
principalmente perché non si sa né cosa dire né come dirla. Per questo, la
prima cosa importante da chiedersi è l'obbiettivo del nostro discorso.

\section{Obiettivo}

Il primo punto di una presentazione è l'\textbf{obiettivo}: qual è lo scopo
della presentazione? Esso può essere differente, come ad esempio l'istruzione
di personale, persuasione di persone o la vendita di prodotti. L'obiettivo è
l'elemento centrale che va analizzato prima di tutto, in quanto fa
cambiare in maniera molto radicale l'approccio che si deve utilizzare.

\paragraph*{Persuasione} L'obiettivo della persuasione in una presentazione è
far cambiare idea a chi mi ascolta, o per esempio far eseguire un'azione che
all'inizio una persona non era incline a fare. In questo caso è necessario
mantenete un alto livello di emotività nell'esposizione, cercando di parlare
con il mid brain, mentre il livello informativo passa in secondo piano e non
deve essere necessariamente elevato.

\paragraph*{Descrittivo} A differenza della convinzione, qui vengono spiegati
per esempio vari passaggi o la natura dei fatti. In questo caso ci si deve
concentrare su di un alto (altissimo a volte) livello informativo, mentre il
livello emotivo è meno importante, e può essere medio-basso.

\paragraph*{Formativo} Per corsi interni o esterni e per condividere conoscenze.
Il livello informativo del discorso è medio-alto, ma con una quantità di emozioni
variabile, in base a chi si sta rivolgendo e dal contesto, anche se in generale
nell'aspetto formativo si ha che il livello emozionale non è necessariamente
altissimo ma avere un'emozionalità media aiuta ad avere un livello minimo di
emozione. Questo tipo di discorso è simile a quello \textbf{descrittivo} ma
solitamente si differenzia per il pubblico: in quello descrittivo si può
supporre che il pubblico abbia una buona conoscenza del tema trattato, cosa
invece che non si può fare in quello formativo.

\paragraph*{Motivazionale} Per caricare il team per un lavoro o per evitare
preoccupazioni interne all'azienda. In quest'ultimo caso l'aspetto emozionale è
tutto, mentre il contenuto informativo può essere anche nullo.

\section{Target}
La seconda cosa importante da tenere in mente è con chi si sta parlando, in
quanto bisogna rapportarsi adeguatamente.

In base al target possono cambiare:
\begin{itemize}
 \item Gestualità
 \item Modo di vestirsi
 \item Lessico
 \item \dots
\end{itemize}

In generale, è importante capire le competenze del pubblico e il suo obiettivo:
come mai il pubblico mi sta ascoltando? Che tipo di gergo devo usare? Un gergo
tecnico a uno molto semplice? Chi è qui cosa è interessato a sentire? Se il
pubblico ha dei timori riguardo ad un certo argomento, è molto meglio
rassicurarlo immediatamente: in questo modo gli si libererà la mente e sarà in
grado di seguire meglio il discorso.

\section{Contesto}

Il \textbf{contesto} in cui si tiene il discorso può influenzare l'esposizione.
Dove, quanto e in che situazione si parlerà?
Tutto ciò fa cambiare il modo in cui bisogna approcciarsi al pubblico.

Se si deve parlare davanti ad un piccolo gruppo di persone è più facile avere
\emph{feedback} da essi e cambiare di conseguenza il discorso, cosa invece
molto più difficile se la platea è ampia. Un'altra cosa da valutare è se si è
i soli a presentare o ci sono altre persone. Nel primo caso è possibile pensare
di dare meno peso alla parte emotiva, nel secondo, al contrario, è necessario
darle un peso differente poiché può essere importante farsi ricordare.

Anche gli aspetti climatici e temporali influiscono sull'efficacia del
discorso: parlare alla platea in situazione di caldo/freddo o subito prima/dopo
i pasti può sminuire l'efficacia del nostro discorso. In questi casi avversi è
quindi meglio, se possibile, diminuire la durata del discorso e aumentare la
carica emotiva.

Solo dopo tutto ciò è possibile passare alla scelta dei contenuti.

\section{Contenuti}

\todo{Ha detto che non entra in dettaglio ora, ma potrebbe farlo nelle prossime
lezioni o più avanti! Slide 13.8}

Per mille motivi, il nostro cervello assorbe bene le storie, piuttosto di
elenchi asettici di funzioni. Quando si presenta un argomento tecnico si può
commettere l'errore di elencare in maniera asettica tutte le caratteristiche,
bombardando l'interlocutore di aspetti rigorosamente tecnici. A meno che il
nostro interlocutore non sia un conoscitore di quell'aspetto tecnico (e quindi
ci sia un discorso tra ``nerd'' del settore) le storie sono il metodo più
efficace per comunicare il prodotto a un venditore.

\section{Business model}

\todo{L'ha saltato in tronco}

\section{Trend, dimensione del mercato e CAGR}

È importante che quando si presenta un progetto, di qualsiasi tipo sia, è
fondamentale presentare i dati \textit{citando le fonti}.
Quando si affrontano delle argomentazioni e lo si fa con forza bisogna citare
sempre le fonti dei dati che si comunicano, contestualizzandoli.

\section{Suggerimenti vari}

Di seguito si elencano una serie di suggerimenti vari.

\subsection{Mostrare energia}

Essere attivi è il primo modo per mantenere il pubblico attivo. È
importante rompere il ritmo: variando la velocità e il tono di voce è
infatti possibile attirare l'attenzione degli ascoltatori. Riguardo al
muoversi nel palco ci sono diverse metodologie e scuole di pensiero. In
definitiva, attirare l'attenzione è in modo per farsi ricordare.

\subsection{Comunicazione}

Trasmettere quello che si vuole dire è più efficace se fatto in maniera
semplice. La \textbf{semplicità} è un fattore fondamentale, soprattutto nella
comunicazione. Nel caso delle startup, dopo un certo numero di pitch si capirà
cosa fa \emph{``brillare gli occhi all'investitore''}: in questo caso si può dare
molto meno peso a tutto il resto e concentrare tutti gli sforzi su quello.

\subsection{Numeri}

La memoria umana ha dei limiti ben conosciuti. Sette (più o meno 2) è il numero
di oggetti che dovrebbero essere gli oggetti presenti nelle slide\footnote{Come
per esempio il numero di elementi in un'interfaccia grafica.}, mentre tre il
numero di concetti che si vuole comunicare con una presentazione e nel caso è
meglio ripeterli tre volte (con parole differenti).

\subsection{Stili di comunicazione}

Nella popolazione esistono diversi tipi di persone, ovvero o quelle uditive o
visive. Le prime danno maggior peso ad un testo, le seconde alle immagini.

Cercare di creare un contatto con il pubblico è porta a diversi vantaggi: è
possibile modificare il proprio intervento in base alle espressioni facciali
dell'interlocutore. Anche guardare negli occhi porta ad avere una maggiore
attenzione da parte dell'ascoltatore. Lo svantaggio è di farsi intimidire.
Evitare i muri tra il presentatore e il pubblico è una buona prassi, come del
resto parlare uno stesso gergo e usare gli stessi argomenti, per creare un
ponte tra me e chi mi sta ascoltando.

Un altro fattore da tenere in considerazione è lo spazio personale da lasciare
alla persona con cui si parla: questo varia molto in base al legame con essa,
dalla sua cultura e dal suo modo di essere.
