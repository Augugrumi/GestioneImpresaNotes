Proseguiamo ora a vedere gli ultimi punti della value proposition.

\paragraph*{Riduzione dei rischi} Per un cliente adottare un nuovo prodotto
implica un rischio (cedimenti tecnologici, malfunzionamento o servizi di
qualità ignota). Ridurre il rischio dovuto alla nuova adozione è importante.
Le value proposition importanti qui sono la riduzione dei rischi lato cliente
ad esempio o il supporto tecnico.

\paragraph*{Accessibilità} Rendere i Prodotti o i Servizi disponibili a fasce
di clienti che precedentemente non potevano accedervi. Banalmente, un esempio
potrebbe essere un ristorante etnico in un luogo in cui non ce ne sono di
quella tipologia.
Questa value proposition è molto ``effimera'', in quanto è facile che nascano
concorrenti, annullando quindi questa value proposition e portando alla
creazione delle altre viste precedentemente.
Da qui l'esigenza di un buon piazzamento di vari negozi/servizi in aree
strategiche.

\paragraph*{Usabilità} Rendere i prodotti più facili da usare può aprire a
fasce di clienti completamente nuove.

\todo{Il prof aveva parlato di USP e ha detto che la si trova nel libro.
Sarebbe figo avere un riassuntino qui.}

\chapter{Public Speaking}

Saper parlare in pubblico è importante, perché permette di vendere meglio il
proprio prodotto, capire quando qualcuno sta cercando di vendere ``fumo'' con
belle parole e riuscir ad ottenere presentazioni pubbliche generalmente
migliori.

\section{Obiettivo}

Il primo punto di una presentazione è \textbf{l'obiettivo}: qual è l'obiettivo
della presentazione? Esso può essere differente, come ad esempio l'istruzione
di personale, persuasione di persone o la vendita di prodotti. Di seguono
vengono esposti in maggior dettaglio.

\paragraph*{Persuasione} L'obiettivo della persuasione in una presentazione è
far cambiare idea a chi mi ascolta, o per esempio far eseguire un'azione che
all'inizio una persona non era incline a fare.

\paragraph*{Descrittivo} A differenza della convinzione, qui vengono spiegati
per esempio vari passaggi o la natura dei fatti

\paragraph*{Formativo} Per corsi interni o esterni

\paragraph*{Motivazionale} Per caricare il team per un lavoro o per evitare
preoccupazioni interne all'azienda\\[0.3cm]

L'obiettivo è l'elemento centrale che va analizzato prima di tutti in quanto fa
cambiare in maniera molto radicale l'approccio che si va ad utilizzare.

Se si ha come obiettivo la persuasione, è necessario esprimere un livello non
troppo alto di informazioni, mentre il discorso che devo fare dev'essere molto
emozionale, in quanto sto tentando di comunicare con il mid brain delle
persone. Per l'obiettivo descrittivo avremo un altissimo contenuto di
informazioni ma basso di informazioni. Per il formativo invece sarà presente un
alto livello di informazioni, ma con una quantità di emozioni variabile, in base
a chi si sta rivolgendo e dal contesto, anche se in generale nell'aspetto
formativo si ha che il livello emozionale non è necessariamente altissimo ma
avere un'emozionalità media aiuta ad avere un livello minimo di emozione.
Nell'ultimo caso, ovvero in una presentazione di tipo motivazionale,
l'emotività è fondamentale, mentre il contenuto informativo può essere anche
nullo.

\section{Target}

È sempre fondamentale avere chiaro con chi si sta parlando, in quanto bisogna
rapportarsi adeguatamente.

In base al target possono cambiare:
\begin{itemize}
 \item Gestualità
 \item Modo di vestirsi
 \item Lessico
 \item \dots
\end{itemize}

In generale, è importante capire anche l'obiettivo dell'\textit{audience}: come
mai il pubblico mi sta ascoltando? Che tipo di gergo devo usare? Un gerco
tecnico a uno molto semplice? Chi è qui cosa è interessato a sentire? Se il
pubblico ha dei timori riguardo ad un certo argomento, è molto meglio
rassicurarlo immediatamente: in questo modo gli si libererà la mente e sarà in
grado di seguire meglio il discorso.

Il \textbf{contesto} è importante. Dove quanto e in che situazione si parlerà?
Tutto ciò fa cambiare il modo in cui bisogna approcciarsi al pubblico.
Presentare da soli è, per esempio, diverso da parlare in gruppo. In generale,
il contesto influenza tutto il discorso e la sua durata.

Finito tutto ciò, si pasa ai \textbf{contenuti} del discorso.

\section{Contenuti}

\todo{Ha detto che non entra in dettaglio ora, ma potrebbe farlo nelle prossime
lezioni o più avanti! Slide 13.8}

Per mille motivi, il nostro cervello assorbe bene le storie, piuttosto di
elenchi asettici di funzioni. Quando si presenta un argomento tecnico si può
commettere l'errore di elencare in maniera asettica tutte le caratteristiche,
bombardando l'interrocutore di aspetti rigorosamente tecnici. A meno che il
nostro interlocutore non sia un conoscitore di quell'aspetto tecnico (e quindi
ci sia un discorso tra ``nerd'' del settore) le storie sono il metodo più
efficace per comunicare il prodotto a un venditore.


\section{Business model}

\todo{L'ha saltato in tronco}

\section{Trend, dimensione del mercato e CAGR}

È importante che quando si presenta un progetto, di qualsiasi tipo sia, è
fondamentale presentare i dati \textit{citando le fonti}.
Quando si affrontano delle argomentazioni e lo si fa con forza bisogna citare
sempre le fonti dei dati che si comunicano, contestualizzandoli.

\section{Suggerimenti vari}

Di seguito si elencano una serie di suggerimenti vari.

\subsection{Mostrare energia}

Essere attivi è il primo modo per mantenere il pubblico attivo. È importante
rompere il ritmo, variando velocità e tono di vocie. Sul muoversi nel palco ci
sono diverse metodologie e scuole di pensiero. In definitiva, attirare
l'attenzione è in modo per farsi ricordare.

\subsection{Comunicazione}

Trasmettere quello che si vuole dire è più efficace se fatto in maniera
semplice. La \textbf{semplicità} è un fattore fondamentale, soprattutto nella
comunicazione.

\subsection{Numeri}

La memoria umana ha dei limiti ben conosciuti. Sette è il numero di oggetti che
dovrebbero essere gli oggetti presenti nelle slide\footnote{Come per esempio
il numero di elementi in un'interfaccia grafica.}, mentre tre dovrebbe essere
la lunghezza degli elenchi puntati. La ripetizione di concetti aiuta la sua
radicazione nella mente degli ascoltatori.

\subsection{Stili di comunicazione}

Nella popolazione esistono diversi tipi di persone, ovvero o quelle uditive o
visive.

Cercare di creare un contatto con il pubblico è porta a diversi vantaggi: è
possibile modificare il proprio intervento in base alle espressioni facciali
dell'interlocutore. Anche guardare negli occhi porta ad avere una maggiore da
parte dell'ascoltatore. Evitare i muri tra il presentatore e il pubblico è una
buona prassi, come del resto parlare uno stesso gergo e usare gli stessi
argomenti, per creare un ponte tra me e chi mi sta ascoltando.
