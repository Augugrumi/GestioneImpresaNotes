Proseguiamo ora a vedere gli ultimi punti della value proposition.

\paragraph*{Riduzione dei rischi} Per un cliente adottare un nuovo prodotto
implica un rischio (cedimenti tecnologici, malfunzionamento o servizi di
qualità ignota). Ridurre il rischio dovuto alla nuova adozione è importante.
Infatti, ogni qualvolta si cambia un prodotto il cliente deve sostenere il
\emph{costo di commutazione}: questo per esempio prevede l'imparare ad usare il
nuovo prodotto, sostituire applicazione presenti con il precedente prodotto e
non presenti nel nuovo (quando per esempio si passa da Apple ad Android) e così
via. Tale costo è ancora più importante se si è un'azienda, poiché va
moltiplicato per il numero di dipendenti. In questo tipo di value proposition
possono risultare importanti:
\begin{itemize}
\item garanzie;
\item supporto o/e servizio clienti;
\item reputazione del marchio;
\item assicurazioni.
\end{itemize}

\paragraph*{Accessibilità} Rendere prodotti o servizi disponibili a fasce
di clienti che precedentemente non potevano accedervi. Un semplice esempio
potrebbe essere l'apertura di un ristorante etnico dove non ce ne sono di
quella tipologia. Questa value proposition può essere molto debole, in quanto
è facile che nascano concorrenti, annullando il vantaggio iniziale. In tal
caso, quindi, è necessario passare ad altri tipi di value proposition.
Anche l'apertura di centri commerciali e il posizionamento di negozi/servizi in
aree strategiche si può basare sull'accessibilità.

\paragraph*{Usabilità} Rendere i prodotti più facili da usare può aprire a
fasce di clienti completamente nuove. Come esempi troviamo i primi iPhone e,
soprattutto, la Nintendo Wii. Quest'ultima ha aperto un mercato totalmente
nuovo, il \emph{family gaming}, semplificando il joystick, creando giochi
che non richiedono un hardware importante e abbassando il costo della
console.

\subsection{USP}

La \textbf{USP} (\textit{Unique Selling Proposition}) si riassume semplicemente 
come una frase ad effetto che deve convincere l'acquirente ad acquistare il 
nostro prodotto piuttosto di quello di un concorrente.
Affinché la USP sia efficace essa deve essere incentrata sulla \textit{reason 
why}, ovvero l’argomentazione o la prova dei fatti impiegata a sostegno della 
promessa fatta dal prodotto ai consumatori\footnote{Definizione presa da 
\url{http://www.glossariomarketing.it/significato/reason-why/}}. Questa 
premessa deve essere chiara e facilmente comprensibile per il mercato a cui si 
rivolge\footnote{La parte sulla USP è stata direttamente riassunta dal libro 
del corso fornito dal professore.}.

\subsection{Customer segment - CS}

Definiscono i diversi gruppi di persone o organizzazioni che il business si
propone di raggiungere e servire.

Agli occhi degli investitori risulta più interessante e apprezzato un prodotto
che mira a una nicchia limitata, ma ben definita, di clienti, piuttosto che un
prodotto di massa.

Individuare con precisione un target permette di capire i suoi veri bisogni.

Bisogna studiare bene il settore, confrontarsi con i possibili clienti ed
essere sempre pronti, se necessario, ad abbandonare l'idea iniziale per
adattarla alle reali necessità del mercato.

\subsection{Canali - CH}

I canali sono i modi in cui la startup comunica e raggiunge il suo segmento
di mercato, veicolano tutte le informazioni e i contatti tra l'azienda e i
clienti (esempio: Internet, tv, giornali, punti vendita, fiere, ecc\dots).

Una volta chiarito il CS, occorre scegliere bene i canali. Quali canali sono
i migliori per i clienti che ho individuato? Che ambienti frequentano, come
si comportano, chi influenza le loro scelte?

Il processo di acquisto si acquista su 5 fasi distinte:

\begin{itemize}

\item \textbf{Consapevolezza}. Bisogna far sapere ai clienti che la
startup/azienda esiste. Di solito si usa la pubblicità tradizionale, tv,
giornali, Internet o fiere. Una delle difficoltà principali delle startup
è quella di essere completamente sconosciuta (o venire ignorata) appena
avviata.

\item \textbf{Valutazione}. I cliente, una volta che hanno conosciuto la
startup/azienda valutano la sua VP. In questa fase entrano in gioco i
cataloghi, i listini, le recensioni, il passaparola (esempio: Booking.com
che trova il suo punto di forza nelle recensioni lasciate dai clienti).

\item \textbf{Acquisto} L'acquisto può avvenire online, fisicamente, tramite
un contratto a domicilio o per posta. L'importante è che sia il meno complicato
possibile (esempio: Amazon 1-click).

\item \textbf{Consegna} Come verrà consegnato il prodotto/servizio? Quali sono
le tempistiche e i rischi? Occorre tenere informato il cliente durante questa
fase (a meno che l'acquisto non sia stato fatto in un negozio fisico e il
prodotto fosse subito disponibile).

\item \textbf{Post vendita} Finché un cliente utilizzerà un prodotto dovrà
essere tenuto aperto un canale di comunicazione con lui (ad esempio per
l'assistenza).

\end{itemize}

\subsection{Customer relationship - CR}

Descrive il tipo di rapporto che un'organizzazione ha con i suoi clienti
(passati, presenti e futuri).

Una relazione può essere\dots

\begin{itemize}

\item basata sull'assistenza personale (prima, durante o dopo il processo di
acquisto).

\item basata sull'assistenza personale dedicata (come la prima, ma in questo
caso ci sarà sempre la stessa persona per un cliente - come nel rapporto
medio/paziente). Questa relazione si basa sulla fiducia.

\item self-service. Questa è una non-relazione: si dà al cliente un libretto
delle istruzioni/cd e si spera che basti a risolvere eventuali dubbi/problemi
(esempio Ikea).

\item self-service + supporto automatico. Qui non c'è relazione con un essere
umano, ma con una macchina/software.

\item di comunità. Se attorno a un prodotto si è creata una community, spesso e
volentieri ci si rivolge ai suoi membri per consigli/suggerimenti (esempio:
gaming online)

\item Co-creazione: alcune organizzazioni si fanno aiutare dai clienti stessi
a individuare e creare valore (esempio: crowdsourcing).

\end{itemize}

La difficoltà è quella di individuare una relazione soddisfacente per il
cliente e che si integri al resto del business model.

\subsection{Key resource - KR}

Le key resource (attività chiave) sono gli asset necessari per far funzionare
un modello di business.
Cosa è necessario per creare il prodotto, farsi conoscere dai clienti e
consegnare loro la VP?
Esistono 4 tipologie di risorse:

\begin{itemize}
 \item \textbf{Risorse fisiche}
 \item \textbf{Risorse umane}
 \item \textbf{Risorse intellettuali}
 \item \textbf{Risorse finanziarie}
\end{itemize}

\subsection{Key activities - KA}

Le key activities trasformano le KR in VP. Sono le attività cruciali che
l'organizzazione deve seguire per far funzionare il modello di business.
Possono essere attività di tipo produttivo (riguardanti la progettazione e
realizzazione di un prodotto), problem solving (risolvere un problema specifico
sottoposta dal cliente), gestione e promozione. Ad esempio, per Facebook una
delle attività chiave è cambiare e ridisegnarsi per mantenere alto l'interesse
dei suoi utenti.

\subsection{Key partnership - KP}

Spesso le startup non generano tutto internamente, ma si inseriscono in filiere
produttive. Le partnership possono servire anche per espandere il proprio
business e raggiungere mercati altrimenti fuori portata.

Una startup non deve considerare il resto del mondo come un nemico: trovare
alleati è un aiuto importante.

Esistono 4 tipi di partnership:

\begin{itemize}
 \item \textbf{Cliente-Fornitore}: questa relazione si basa sulla
fedeltà dei clienti acquisiti.
 \item \textbf{Alleanza strategica tra non-competitor}: relazione
tra due aziende non competitor, che possono però mirare alla stessa
nicchia di clienti.
 \item \textbf{Cooperazione}: alleanza tra competitor che può avvenire
per vari motivi (suddivisione dei rischi, definire un nuovo standard,
attaccare un leader di mercato\dots)
 \item \textbf{Joint venture}: accordo puramente finanziario, ad esempio
tra una startup e Venture Capital o Business Angel.
\end{itemize}

\subsection{Cost structure - C\$}

La cost structure descrive tutti i costi sostenuti per operare in un modello di
business. Creare una VP, consegnarla ai clienti e mantenere con loro una
relazione ha un costo di cui deve essere tenuto conto quando si hanno gli
investimenti.

Esistono due tipi di business model che possono aiutare a capire come
suddividere le spese:

\begin{itemize}
 \item Modelli \textbf{cost driven}: business pensati per offrire ai clienti
la VP con il costo più basso possibile. Per questi modelli sono importanti
il mantenimento di strutture snelle, l'autoproduzione e l'outsourcing.
 \item Modelli \textbf{value driven}: business in cui la qualità del prodotto
è più importante che mantenere un prezzo basso (ad esempio, hotel a 5 stelle).
\end{itemize}

\subsection{Revenue stream - R\$}

Individuare il pricing corretto per un prodotto/servizio è un'attività
cruciale. Un approccio comune per business consolidati è partire dal costo di
produzione della VP e aggiungerci il margine che si vuole ottenere.
Per business che devono decollare si parte dai clienti: quanto sono disposti a
pagare per la VP offerta? A quale aspetto sono interessati? La risposta
a questa domanda farà conoscere i flussi di incassi per ogni segmento
(di clienti) che dovranno coprire i costi e generare margine per permettere
alla start up di guadagnare.

\subsection{Il percorso del valore}

Non sempre tutti i blocchi del business model sono critici, occorre avere
chiaro quali sono quelli primari e quali quelli secondari. Si deve capire come
sono collegati e qual à il percorso che genera valore. Una configurazione
comune parte dai costi passando per i blocchi produttivi che generano proposta
di valore, che viene consegnata ai clienti, che pagando, generano ricavi.

Attenzione, questo percorso non stabilisce in che ordine vanno compilati i
vari blocchi del Business Model, non ha nulla a che fare con il suo design!

\subsection{Business Model design}

Come ideare un Business Model? La prima cosa da fare è studiare molto bene il
settore in cui si vuole lavorare fino a conoscerlo alla perfezione.
Bisogna capire chi sono i players, grandi e piccoli. Studiando il loro
percorso nel settore e gli eventuali errori che hanno commesso si può
evitare di fare gli stessi sbagli e risparmiare del tempo. Bisogna anche
individuare quali sono i trend sociali e tecnologici del settore.
Sono tre gli aspetti su cui focalizzarsi in questa fase iniziale:

\begin{itemize}

\item \textbf{I competitor} Chi è stato/è presente nel mercato?
Cosa ha fatto di giusto? Che eventuali errori ha commesso?

\item \textbf{Le tecnologie} Quali tecnologie possono essere utili al progetto?
Qual è il loro costo?

\item \textbf{I trend} Cosa succederà tra qualche anno, quali saranno gli
sviluppi possibili? Verso dove stanno puntando gli aspetti legislativi e
normativi?

Una volta studiati questi elementi si potrà ideare qualcosa di nuovo,
generando molte idee e identificando le migliori.
Le idee vanno analizzate, valutando pregi e difetti, originalità e
fattibilità e la passione che esse suscitano.

Quando si hanno una manciata di idee potenzialmente interessanti vanno
costruiti i primi Business Model.

Per ogni blocco dei Business Model realizzati ci si deve chiedere quali
sono i suoi punti di forza e le debolezze, quali sono i requisiti
necessari, quali gli aspetti innovativi, cosa manca nel mercato che si
possiede?

La selezione finale per la scelta di un'unica idea va fatta in base
(di nuovo) a forze, debolezze, fattibilità tecnologica e sostenibilità
economica.

\end{itemize}

\chapter{Public Speaking}

Saper parlare in pubblico è importante, perché permette di vendere meglio il
proprio prodotto ma anche per capire quando qualcuno sta cercando di vendere
``fumo'' con belle parole. Probabilmente le prime esperienze non sono positive,
principalmente perché non si sa né cosa dire né come dirla. Per questo, la
prima cosa importante da chiedersi è l'obbiettivo del nostro discorso.

\section{Obiettivo}

Il primo punto di una presentazione è l'\textbf{obiettivo}: qual è lo scopo
della presentazione? Esso può essere differente, come ad esempio l'istruzione
di personale, persuasione di persone o la vendita di prodotti. L'obiettivo è
l'elemento centrale che va analizzato prima di tutto, in quanto fa
cambiare in maniera molto radicale l'approccio che si deve utilizzare.

\paragraph*{Persuasione} L'obiettivo della persuasione in una presentazione è
far cambiare idea a chi mi ascolta, o per esempio far eseguire un'azione che
all'inizio una persona non era incline a fare. In questo caso è necessario
mantenete un alto livello di emotività nell'esposizione, cercando di parlare
con il mid brain, mentre il livello informativo passa in secondo piano e non
deve essere necessariamente elevato.

\paragraph*{Descrittivo} A differenza della convinzione, qui vengono spiegati
per esempio vari passaggi o la natura dei fatti. In questo caso ci si deve
concentrare su di un alto (altissimo a volte) livello informativo, mentre il
livello emotivo è meno importante, e può essere medio-basso.

\paragraph*{Formativo} Per corsi interni o esterni e per condividere conoscenze.
Il livello informativo del discorso è medio-alto, ma con una quantità di emozioni
variabile, in base a chi si sta rivolgendo e dal contesto, anche se in generale
nell'aspetto formativo si ha che il livello emozionale non è necessariamente
altissimo ma avere un'emozionalità media aiuta ad avere un livello minimo di
emozione. Questo tipo di discorso è simile a quello \textbf{descrittivo} ma
solitamente si differenzia per il pubblico: in quello descrittivo si può
supporre che il pubblico abbia una buona conoscenza del tema trattato, cosa
invece che non si può fare in quello formativo.

\paragraph*{Motivazionale} Per caricare il team per un lavoro o per evitare
preoccupazioni interne all'azienda. In quest'ultimo caso l'aspetto emozionale è
tutto, mentre il contenuto informativo può essere anche nullo.

\section{Target}
La seconda cosa importante da tenere in mente è con chi si sta parlando, in
quanto bisogna rapportarsi adeguatamente.

In base al target possono cambiare:
\begin{itemize}
 \item Gestualità
 \item Modo di vestirsi
 \item Lessico
 \item \dots
\end{itemize}

In generale, è importante capire le competenze del pubblico e il suo obiettivo:
come mai il pubblico mi sta ascoltando? Che tipo di gergo devo usare? Un gergo
tecnico a uno molto semplice? Chi è qui cosa è interessato a sentire? Se il
pubblico ha dei timori riguardo ad un certo argomento, è molto meglio
rassicurarlo immediatamente: in questo modo gli si libererà la mente e sarà in
grado di seguire meglio il discorso.

\section{Contesto}

Il \textbf{contesto} in cui si tiene il discorso può influenzare l'esposizione.
Dove, quanto e in che situazione si parlerà?
Tutto ciò fa cambiare il modo in cui bisogna approcciarsi al pubblico.

Se si deve parlare davanti ad un piccolo gruppo di persone è più facile avere
\emph{feedback} da essi e cambiare di conseguenza il discorso, cosa invece
molto più difficile se la platea è ampia. Un'altra cosa da valutare è se si è
i soli a presentare o ci sono altre persone. Nel primo caso è possibile pensare
di dare meno peso alla parte emotiva, nel secondo, al contrario, è necessario
darle un peso differente poiché può essere importante farsi ricordare.

Anche gli aspetti climatici e temporali influiscono sull'efficacia del
discorso: parlare alla platea in situazione di caldo/freddo o subito prima/dopo
i pasti può sminuire l'efficacia del nostro discorso. In questi casi avversi è
quindi meglio, se possibile, diminuire la durata del discorso e aumentare la
carica emotiva.

Solo dopo tutto ciò è possibile passare alla scelta dei contenuti.

\section{Contenuti}

\todo{Ha detto che non entra in dettaglio ora, ma potrebbe farlo nelle prossime
lezioni o più avanti! Slide 13.8}

Per mille motivi, il nostro cervello assorbe bene le storie, piuttosto di
elenchi asettici di funzioni. Quando si presenta un argomento tecnico si può
commettere l'errore di elencare in maniera asettica tutte le caratteristiche,
bombardando l'interlocutore di aspetti rigorosamente tecnici. A meno che il
nostro interlocutore non sia un conoscitore di quell'aspetto tecnico (e quindi
ci sia un discorso tra ``nerd'' del settore) le storie sono il metodo più
efficace per comunicare il prodotto a un venditore.

\section{Business model}

\todo{L'ha saltato in tronco}

\section{Trend, dimensione del mercato e CAGR}

È importante che quando si presenta un progetto, di qualsiasi tipo sia, è
fondamentale presentare i dati \textit{citando le fonti}.
Quando si affrontano delle argomentazioni e lo si fa con forza bisogna citare
sempre le fonti dei dati che si comunicano, contestualizzandoli.

\section{Suggerimenti vari}

Di seguito si elencano una serie di suggerimenti vari.

\subsection{Mostrare energia}

Essere attivi è il primo modo per mantenere il pubblico attivo. È
importante rompere il ritmo: variando la velocità e il tono di voce è
infatti possibile attirare l'attenzione degli ascoltatori. Riguardo al
muoversi nel palco ci sono diverse metodologie e scuole di pensiero. In
definitiva, attirare l'attenzione è in modo per farsi ricordare.

\subsection{Comunicazione}

Trasmettere quello che si vuole dire è più efficace se fatto in maniera
semplice. La \textbf{semplicità} è un fattore fondamentale, soprattutto nella
comunicazione. Nel caso delle startup, dopo un certo numero di pitch si capirà
cosa fa \emph{``brillare gli occhi all'investitore''}: in questo caso si può dare
molto meno peso a tutto il resto e concentrare tutti gli sforzi su quello.

\subsection{Numeri}

La memoria umana ha dei limiti ben conosciuti. Sette (più o meno 2) è il numero
di oggetti che dovrebbero essere gli oggetti presenti nelle slide\footnote{Come
per esempio il numero di elementi in un'interfaccia grafica.}, mentre tre il
numero di concetti che si vuole comunicare con una presentazione e nel caso è
meglio ripeterli tre volte (con parole differenti).

\subsection{Stili di comunicazione}

Nella popolazione esistono diversi tipi di persone, ovvero o quelle uditive o
visive. Le prime danno maggior peso ad un testo, le seconde alle immagini.

Cercare di creare un contatto con il pubblico è porta a diversi vantaggi: è
possibile modificare il proprio intervento in base alle espressioni facciali
dell'interlocutore. Anche guardare negli occhi porta ad avere una maggiore
attenzione da parte dell'ascoltatore. Lo svantaggio è di farsi intimidire.
Evitare i muri tra il presentatore e il pubblico è una buona prassi, come del
resto parlare uno stesso gergo e usare gli stessi argomenti, per creare un
ponte tra me e chi mi sta ascoltando.

Un altro fattore da tenere in considerazione è lo spazio personale da lasciare
alla persona con cui si parla: questo varia molto in base al legame con essa,
dalla sua cultura e dal suo modo di essere.
