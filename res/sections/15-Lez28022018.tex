\section{Dal consumo al servizio} \todo{Boh? L'ha fatto ad inizio lezione,
forse in risposta a una domanda}

Negli ultimi anni si sta passando da un'economia definita di \textit{consumo} a
una di \textit{servizio}, basata sulle condivisione delle risorse.
Nell'economia del possesso il bene viene comprato perché visto come una
necessità, ma se di fatto prendiamo come esempio un'automobile ci sono molte ore
di inutilizzo, nelle quali il mezzo potrebbe essere utilizzato da altre persone.
Questa soluzione è raggiungibile tramite, per esempio, un servizio in
abbonamento, in cui si paga solamente per il tempo di effettivo utilizzo 
dell'auto.


\chapter{Governance}

Per governance si intende chi governa un'azienda.

\section{Organigramma aziendale}
L'organigramma aziendale definisce le persone che operano all'interno
dell'azienda e il loro ruolo. La struttura aziendale può essere decomposta in
macrolivelli: in una struttura piramidale, alla base si hanno gli
\textbf{operativi} (impiegati, dipendenti, operai, ecc\dots). Al livello
successivo si hanno i \textbf{quadri intermedi}: stanno sotto la dirigenza e al
management aziendale. Di solito sono presenti nelle grandi aziende, dove sono
necessari delle figure responsabili di gruppi e aree aziendali, ma che non
fanno ancora parte della dirigenza. La dirigenza invece è costituita da soci e
imprenditori, mentre nelle aziende strutturate ci sono dirigenti che non hanno
quote di partecipazione alla società, ma si limitano a gestire le operazioni
aziendali. Al vertice della piramide troviamo gli imprenditori e il consiglio
di amministrazione.

\paragraph*{Marketing} Lo scopo del reparto marketing è quella di creare una
connessione tra l'azienda (in particolare, il suo prodotto) e i clienti: questo
è molto diverso dal concetto di vendita.\todo{Il prof ha particolarmente
marcato questo passaggio, definendolo importante} Nelle PMI\footnote{Piccole e
Medie Imprese} italiane, vendita e marketing sono molto spesso mescolate, e
questo non va bene.

Il marketing può essere diviso in due aree di competenza: quello
\textit{operativo} (spesso definito solamente come marketing), che comprende i
biglietti da visita, l'immagine dell'azienda, il marketing della comunicazione,
slogan e pubblicità, e quello \textit{strategico}, che precede quello
operativo, nel quale è necessario definire l'obiettivo della pubblicità e ciò
che va comunicato al pubblico (perché un cliente deve comprare il mio prodotto
rispetto a quello del mio competitor) e a chi è diretto il messaggio (non a
tutti potrebbe interessare un determinato prodotto, devo sapere a chi sto
vendendo). È importante quindi avere delle strategie chiare e ben definite.
Solo successivamente è possibile passare al marketing operativo.

Il marketing strategico lavora in maniera molto stretta col reparto produzione,
in quanto bisogna pensare alle caratteristiche del prodotto che andrà venduto.
Il marketing operativo riceve queste informazioni e trova i modi operativi più
efficaci per presentare il prodotto alla clientela. In tutto ciò non si parla
mai direttamente di vendita, in quanto, come già detto, è onere del reparto
vendite.

\subsection{Dirigenza}

\paragraph*{CEO/Amministratore delegato} È la figura operativa apicale nella
gerarchia aziendale (chiamato CEO\footnote{Chief Executive Officer} in
inglese). È responsabile di tutte le attività aziendali, ed è il legale
rappresentante della società\footnote{Ovvero è la figura a cui pendono i capi
d'imputazione in caso di un processo.}. Controlla e prende tutte le decisione
operative sulla società.

Essendo la persona in capo all'azienda, prende le decisioni per tutte le
attività sia ordinarie che straordinarie. In società piccole,
amministratore delegato e socio sono la stessa persona. Nelle aziende più
strutturate dove non è presente un singolo socio, c'è un consiglio di
amministrazione (di cui lui fa parte l'amministratore delegato).
Ma fino a che punto può decidere? I limiti vengono decisi dal board stesso (in
caso esista).

\subparagraph*{Differenza con amministratore unico} Si ha un amministratore
unico quando non si ha un board, e quindi si è l'unico amministratore.

\paragraph*{Consiglio di amministrazione} Il \textit{board of directors} di
un'azienda è composto dai suoi soci. Il board ha un suo presidente (più di
rappresentanza) e dei membri, che eleggono l'amministratore delegato.

Alcune decisioni vengono prese dal board in quanto non è possibile farle
dall'amministratore delegato (per esempio contratti con costi sopra una certa
soglia). Nel board avviene solitamente una votazione che decide se dare o meno al CEO un determinato potere.

In aziende molto grosse il board non è completamente sovrapponibile alla
compagine sociale (si pensi ad esempio a società quotate in borsa). Le azioni
corrispondono a quote societarie dell'azienda.

\chapter{Filiera produttiva}

\todo{Dare un titolo a questa parte perché è completamente scollegata dalle
altre, manca contesto.}

Le aziende non sono delle isole: hanno dei prodotti che fanno un percorso 
attraverso vari fornitori e compratori: tale percorso è definito come
\textbf{filiera produttiva}.

I \textbf{fornitori} sono coloro che offrono prodotti/servizi necessari per
poter vendere il mio prodotto. Per i fornitori un'azienda che acquista i propri
prodotti è un cliente. Le società hanno multipli fornitori che poi
vendono ai clienti.\todo{Non ho capito il senso di questa frase}
Ogni filiera produttiva ``vive'' vicino ad altre filiere,
che producono prodotti differenti e che possono interconnettersi supportando i
prodotti che vendono.

Prendendo ad esempio un forno, egli prende la farina e la trasforma in pane. La
farina arriva dal fornitore, e il pane viene venduto ai clienti. I sacchetti
per il pane arrivano da dei fornitori, ma da una differente filiera produttiva!
Le filiere produttive quindi possono interlacciarsi e possono essere molto
lunghe.

Filiere produttive lunghe causano un aumento dei costi (in quanto ogni anello
della catena deve avere dei margini di guadagno su quello che offre). Si hanno
degli aumenti nei costi anche quando un determinato elemento è poco: ciò
aumenta il prezzo, causando di conseguenza in incremento del costo per il
cliente finale. Aumenti del prezzo di un prodotto a monte della catena ha
effetti disastrosi quando si arriva a valle.

\section{Tipologie di clienti}

Esistono diverse tipologie di aziende:
\begin{itemize}
  \item \textbf{B2C: Business to Consumer}: i prodotti vengono direttamente
  venduti al dettaglio alla clientela. I clienti solitamente sono un numero
  enormemente elevato e di cui conosco solo certe caratteristiche (età, sesso,
  luogo di vendita). In questo caso è quasi impossibile che un venditore di una
  specifica azienda venga a proporti il proprio prodotto. Le strategie di
  vendita avvengono in maniera diversa, sfruttando per esempio la pubblicità al
  \textit{mass market}, senza avere un contatto diretto con la singola persona.
  Tali tecniche hanno di solito una bassa retention\footnote{Il ritorno di
  quell'azione.}, ma essendo applicata a un alto numero di potenziali
  acquirenti finali fruttano comunque un certo guadagno.

  \item \textbf{B2B: Business to Business}: i prodotti vengono venduti ad altre
  aziende. I clienti sono molti meno e, quindi, meglio conosciuti. In questo
  caso è possibile avere venditori che incontrano altre aziende per vendere il
  proprio prodotto, con un approccio \textit{one to one}, diverso dal
  \textit{mass market}.

  \item \textbf{B2B2C}: Sono aziende che vendono un prodotto che arriva a
  clienti finali, ma non hanno il controllo diretto su di essi.

  \item \textbf{B2G: Business to Government}: i prodotti vengono venduti ai
  governi. In questo caso si ha un'alta complessità legale e burocratica in e
  l'azienda lavora tramite assegnazione di bandi pubblici e vincite di gare
  d'appalto. Anche i metodi di pagamento sono diversi dai precedenti, hanno
  tempi più lunghi e modalità che cambiano da paese a paese.
\end{itemize}

Le diverse tipologie di clienti causano un cambiamento nell'approccio che si ha
per la vendita del prodotto.
