\section{Dal consumo al servizio} \todo{Boh? L'ha fatto ad inizio lezione,
forse in risposta a una domanda}

Negli ultimi anni si sta passando da un'economia definita di
\mKeywordit{consumo} a una di \mKeywordit{servizio}, basata sulle condivisione
delle risorse.
Nell'economia del possesso il bene viene comprato perché visto come una
necessità, ma se di fatto prendiamo come esempio un'automobile ci sono molte ore
di inutilizzo, nelle quali il mezzo potrebbe essere utilizzato da altre persone.
Questa soluzione è raggiungibile tramite, per esempio, un servizio in
abbonamento, in cui si paga solamente per il tempo di effettivo utilizzo
dell'auto.

L'economia del possesso ha anche un altro scopo, che è quello di dare agli altri
una certa immagine di sé. Riprendendo l'esempio dell'automobile, spesso la gente
compra macchine che vanno oltre le loro necessità, solo come simbolo di
ricchezza.


\chapter{Governance}

Per governance si intende chi governa un'azienda.

\section{Organigramma aziendale}
\begin{definition}[Organigramma aziendale]
L'organigramma aziendale definisce le persone che operano all'interno
dell'azienda e il loro ruolo.
\end{definition}

\noindent La struttura aziendale può essere decomposta in macrolivelli: in una
struttura piramidale, alla base si hanno gli \textbf{operativi} (impiegati,
dipendenti, operai, ecc\dots). Al livello successivo si hanno i \textbf{quadri
intermedi}: stanno sotto la dirigenza e al management aziendale. Di solito sono
presenti nelle grandi aziende, dove sono necessari delle figure responsabili di
gruppi e aree aziendali, ma che non fanno ancora parte della dirigenza. La
dirigenza invece è costituita da soci e imprenditori, mentre nelle aziende
strutturate ci sono dirigenti che non hanno quote di partecipazione alla
società, ma si limitano a gestire le operazioni aziendali (management). Al
vertice della piramide troviamo gli imprenditori e il consiglio di
amministrazione (board).

\paragraph*{Produzione} È il reparto che produce il bene. In alcuni casi
(soprattutto nelle PMI\footnote{Piccole e Medie Imprese}), ingloba il
reparto di R\&D`\footnote{Abbreviazione per ricerca e sviluppo.}.

\paragraph*{Marketing}
\begin{definition}[Marketing]
Lo scopo del reparto marketing è quella di creare una connessione tra l'azienda
(in particolare, il suo prodotto) e i clienti.
\end{definition}

\noindent Il marketing è \textbf{molto diverso} dal concetto di vendita: nelle
PMI italiane, vendita e marketing sono molto spesso mescolate, e questo non va
bene.

Il marketing può essere diviso in due aree di competenza: quello
\mKeywordit{operativo} (spesso definito solamente come marketing), che comprende
i biglietti da visita, l'immagine dell'azienda, il marketing della
comunicazione, slogan e pubblicità, e quello \mKeywordit{strategico}, che
precede quello operativo, nel quale è necessario definire l'obiettivo della
pubblicità e ciò che va comunicato al pubblico (perché un cliente deve comprare
il mio prodotto rispetto a quello del mio competitor) e a chi è diretto il
messaggio (non a tutti potrebbe interessare un determinato prodotto, devo
sapere a chi sto vendendo). È importante quindi avere delle strategie chiare e
ben definite. Solo successivamente è possibile passare al marketing operativo.

Il marketing strategico lavora in maniera molto stretta col reparto produzione,
in quanto bisogna pensare alle caratteristiche del prodotto che andrà venduto.
Il marketing operativo riceve queste informazioni e trova i modi operativi più
efficaci per presentare il prodotto alla clientela. In tutto ciò non si parla
mai direttamente di vendita, in quanto, come già detto, è onere del reparto
vendite.

\subsection{Dirigenza}

\paragraph*{CEO/Amministratore delegato} È la \mKeyword{figura operativa
apicale} nella gerarchia aziendale (chiamato CEO\footnote{Chief Executive
Officer} in inglese). È responsabile di tutte le attività aziendali, ed è il
legale rappresentante della società\footnote{Ovvero è la figura a cui pendono i
capi d'imputazione in caso di un processo.}. Controlla e prende tutte le
decisioni operative su di essa, ma anche di quelle strategiche a lungo
termine.

Essendo la persona in capo all'azienda, prende le decisioni per tutte le
attività sia ordinarie che straordinarie. In società piccole,
amministratore delegato e proprietario sono solitamente la stessa persona. Nelle
aziende più strutturate dove non è presente un singolo socio, c'è un
consiglio di amministrazione (di cui fa parte l'amministratore delegato).
Ma fino a che punto può decidere? I limiti vengono decisi dal board stesso (in
caso esista), e stabiliti nel contratto del CEO. In caso di violazione dei
limiti stabiliti, ogni contratto siglato dal CEO è nullo.

\subparagraph*{Differenza con amministratore unico} Si ha un amministratore
unico quando non si ha un board, e quindi si è l'unico amministratore.

\paragraph*{Consiglio di amministrazione} Il \textit{board of directors} di
un'azienda è composto dai suoi soci. Il board ha un suo presidente (più di
rappresentanza) e dei membri, che \mKeyword{eleggono l'amministratore
delegato}.

Alcune decisioni vengono prese dal board in quanto non è possibile farle
dall'amministratore delegato (per esempio contratti con costi sopra una certa
soglia). Nel board avviene solitamente una votazione che decide se dare o meno
al CEO un determinato potere.

In aziende molto grosse il board non è completamente sovrapponibile alla
compagine sociale (si pensi ad esempio a società quotate in borsa). Le azioni
corrispondono a quote societarie dell'azienda.

\chapter{Filiera produttiva}

\section{Fornitore}

Le aziende non sono delle isole: hanno dei prodotti che fanno un percorso
attraverso vari fornitori e compratori: tale percorso è definito come
\textbf{filiera produttiva}.

\begin{definition}[Fornitore]
In un'azienda, il fornitore è colui che offre prodotti/servizi necessari per
poter vendere il prodotto finale.
\end{definition}

\noindent Per i fornitori un'azienda che acquista i propri prodotti è un
cliente. \mKeyword{Le società hanno multipli fornitori} che poi vendono ai
clienti. Ogni filiera produttiva ``vive'' vicino ad altre filiere, che producono
prodotti differenti e che possono interconnettersi supportando i prodotti che
vendono.

\begin{example}[Forno]
Prendendo ad esempio un forno, esso prende la farina e la trasforma in pane. La
farina arriva dal fornitore, e il pane viene venduto ai clienti. I sacchetti
per il pane arrivano da dei fornitori, ma da una differente filiera produttiva!
Questi ultimi sono chiamati \textbf{partner}.
\end{example}

\noindent Le filiere produttive quindi possono interlacciarsi e possono essere
molto lunghe.

Filiere produttive lunghe causano un aumento dei costi (in quanto ogni anello
della catena deve avere dei margini di guadagno su quello che offre). Si hanno
degli aumenti nei costi anche quando un determinato elemento è poco: ciò
aumenta il prezzo, causando di conseguenza un incremento del costo per il
cliente finale. Aumenti del prezzo di un prodotto a monte della catena ha
effetti disastrosi quando si arriva a valle. Un altro problema delle filiere
lunghe è l'impatto di una crisi. Più lunga è la catena più un momento di crisi
impatta molte realtà diverse, interessando non solo la catena stessa, ma anche
eventuali filiere partner.

\section{Tipologie di clienti}

Esistono diverse tipologie di aziende:
\begin{itemize}
  \item \textbf{B2C: Business to Consumer}: i prodotti vengono direttamente
  venduti al dettaglio alla clientela. I clienti solitamente sono un numero
  enormemente elevato e di cui conosco solo certe caratteristiche (età, sesso,
  luogo di vendita). In questo caso è quasi impossibile che un venditore di una
  specifica azienda venga a proporti il proprio prodotto. Le strategie di
  vendita avvengono in maniera diversa, sfruttando per esempio la pubblicità al
  \mKeywordit{mass market}, senza avere un contatto diretto con la singola
  persona.
  Tali tecniche hanno di solito una bassa redemption\footnote{Il ritorno di
  quell'azione.}, ma essendo applicate a un alto numero di potenziali
  acquirenti finali fruttano comunque un certo guadagno.

  \item \textbf{B2B: Business to Business}: i prodotti vengono venduti ad altre
  aziende. I clienti sono molti meno e, quindi, meglio conosciuti. In questo
  caso è possibile avere venditori che incontrano altre aziende per vendere il
  proprio prodotto, con un approccio \mKeywordit{one to one}, diverso dal
  \textit{mass market}. In questo caso le pubblicità possono essere più mirate
  (es cataloghi specializzati). Il ciclo di vendita è più lungo rispetto al B2C
  (misurabile in mesi), e le vendite vengono solitamente siglate sotto forma di
  contratti.

  \item \textbf{B2B2C}: Sono aziende che vendono un prodotto che arriva a
  clienti finali, ma non hanno il controllo diretto su di essi.

  \item \textbf{B2G: Business to Government}: i prodotti vengono venduti ai
  governi. In questo caso si ha un'\mKeyword{alta complessità legale e
  burocratica} e l'azienda lavora tramite assegnazione di bandi pubblici e
  vincite di gare d'appalto. Anche i metodi di pagamento sono diversi dai
  precedenti, hanno tempi più lunghi e modalità che cambiano da paese a paese.
\end{itemize}

Le diverse tipologie di clienti causano un cambiamento nell'approccio che si ha
per la vendita del prodotto.
