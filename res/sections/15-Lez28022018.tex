\section{Dal consumo al servizio} \todo{Boh? L'ha fatto ad inizio lezione,
forse in risposta a una domanda}

Negli ultimi anni si sta passando da un'economia definita di \textit{consumo} a
una di \textit{servizio}, basata sulle condivisione delle risorse.
Nell'economia del possesso il bene viene comprato perché lo vedo come una
necessità, ma per esempio per un'auto si hanno un sacco di ore di inutilizzo
dopo il trasporto, dove il mezzo potrebbe essere utilizzato da altre persone.
Questa soluzione la si può raggiungere tramite, per esempio, un servizio in
abbonamento, in cui si paga solamente per l'uso dell'auto.


\chapter{Governance}

Per governance si intende chi governa un'azienda.

\section{Organigramma aziendale}
Organigramma aziendale: definiscono chi sono e dove operano le persone
all'interno dell'azienda. La struttura aziendale può essere divisa in
macrolivelli: in una struttura piramidale, alla base si hanno gli operativi
(impiegati, dipendenti, operai, ecc\dots). Al livello successivo si hanno i
Quadri intermedi: stanno sotto la dirigenza e al management aziendale. Di solito
sono presenti nelle grandi aziende, dove c'è bisogno di responsabili di gruppi
e aree aziendale, ma che non fanno ancora parte della dirigenza. La dirigenza
invece è costituita da soci e imprenditori, mentre nelle aziende strutturate ci
sono dirigenti che non hanno quote di partecipazione alla società, ma si
limitano a gestire le operazioni aziendali. Al vertice della piramide troviamo
gli imprenditori e il consiglio di amministrazione.

\paragraph*{Marketing} Lo scopo del reparto marketing è quella di creare la
connessione tra l'azienda (il suo prodotto) e i clienti: questo è molto diverso
dal concetto di vendita.\todo{Il prof ha particolarmente marcato questo
passaggio, definendolo importante} Nelle PMI italiane, vendita e marketing sono
molto spesso mescolate, e questo non va bene.

Il marketing può essere diviso in due aree di competenza: quello operativo (che
viene spesso definito come marketing e basta, ovvero i biglietti da visita, la
sua immagine, il marketing della comunicazione, slogan e pubblicità) e
quello strategico, che avviene prima di quello operativo, dove bisogna pensare
per esempio all'obiettivo della pubblicità, cosa va comunicato (perché un
cliente deve comprare il mio prodotto rispetto a quello del mio competitor) e a
chi va comunicato (a non tutti potrebbe interessare, devo sapere a chi sto
vendendo). È importante quindi avere delle strategie chiare e ben pensate. Solo
dopo che sono stati eseguiti questi passaggi è possibile passare al marketing
operativo.

Il marketing strategico lavora in maniera molto stretta col reparto produzione,
in quanto bisogna pensare alle caratteristiche del prodotto che andrà venduto.
Il marketing operativo riceve queste informazioni e trova i modi operativi più
efficaci per presentare il prodotto alla clientela. In tutto ciò non si parla
mai direttamente di vendita, in quanto, come già detto, è onere del reparto
vendite.

\subsection{Dirigenza}

\paragraph*{CEO/Amministratore delegato} È la figura operativa apicale nella
gerarchia aziendale (chiamato CEO in inglese). È responsabile di tutte le
attività aziendali, ed è il legale rappresentante della società\footnote{Ovvero
è la figura a cui pendono i capi d'imputazione in caso di un processo.}.
Controlla e prende tutte le decisione operative sulla società.

Essendo la persona in capo all'azienda, prende le decisioni per tutte le
attività ordinarie, ma anche per quelle straordinarie. In società piccole,
amministratore delegato e socio sono la stessa persona. Nelle aziende più
strutturate non c'è un unico socio, ed è presente un consiglio di
amministrazione (di cui lui fa parte).
Ma fino a che punto può decidere? I limiti vengono decisi dal board stesso (in
caso esista).

\subparagraph*{Differenza con amministratore unico} Si ha un amministratore
unico quando non si ha un board, e quindi si è l'unico amministratore.

\paragraph*{Consiglio di amministrazione} Il \textit{board of directors} di
un'azienda è composto dai suoi soci. Il board ha un suo presidente (più di
rappresentanza) e dei membri, che eleggono l'amministratore delegato.

Alcune decisioni vengono prese dal board in quanto non è possibile farle
dall'amministratore delegato (per esempio contratti con costi sopra una certa
soglia). Nel board avviene solitamente una votazione che decide se dare il
potere al CEO di dare questo potere o meno.

In aziende molto grosse il board non è completamente sovrapponibile alla
compagine sociale (si pensi ad esempio a società quotate in borsa). Le azioni
corrispondono a quote societarie dell'azienda.

\chapter{Filiera produttiva}

\todo{Dare un titolo a questa parte perché è completamente scollegata dalle
altre, manca contesto.}

Le aziende non sono delle isole: hanno dei prodotti che fanno un percorso tra
vari compratori e fornitori: questa viene definita come filiera produttiva.

I fornitori sono coloro che vendono prodotti/servizi necessari per poter
vendere il mio prodotto. Per i fornitori un'azienda che acquista quello che
offre vengono detti clienti. Le aziende hanno multipli fornitori che poi
vendono ai clienti. Ogni filiera produttiva ``vive'' vicino ad altre filiere,
che producono altri prodotti e che possono interconnettersi supportando i
prodotti che vendono.

Prendendo ad esempio un forno, egli prende la farina e la trasforma in pane. La
farina arriva dal fornitore, e il pane viene venduto ai clienti. I sacchetti
per il pane arrivano da dei fornitori, ma da un'altra filiera produttiva! Le
filiere produttive quindi possono interlacciarsi, e possono essere molto lunghe.

Filiere produttive lunghe causano un aumento dei costi (in quanto ogni azienda
facente parte di questa catena deve avere dei margini di guadagno su quello che
offre). Si hanno problemi anche quando un determinato elemento è poco, in
quanto il prezzo aumenta, causando un aumento del costo a valle per il cliente
finale. Aumenti del prezzo di un prodotto a monte della catena ha effetti
disastrosi quando si arriva a valle.

\section{Tipologie di clienti}

Esistono diverse tipologie di aziende:
\begin{itemize}
 \item B2C: Business to Consumer: i prodotti vengono direttamente venduti al
dettaglio alla clientela, che è solitamente un numero enormemente elevato e di
cui conosco solo certe caratteristiche (età, sesso, luogo di vendita). In
questo caso è quasi impossibile vedere un venditore di una specifica azienda che
viene a proporre il proprio prodotto. Le strategie di vendita avvengono in
maniera diversa (tramite per esempio la pubblicità al \textit{mass market},
senza avere contatto diretto con la singola persona), e hanno di solito una
bassa retention\footnote{Il ritorno di quell'azione.}, ma essendo applicata a
un alto numero di potenziali acquirenti finali fruttano comunque un certo
guadagno.
 \item B2B: Business to Business i prodotti vengono venduti ad altre aziende. I
clienti sono molti meno, ma sono meglio conosciuti. È possibile mandare un
venditore per cercare di vendere il prodotto, con un approccio \textit{one to
one}, diversamente dal \textit{mass market}.
 \item B2B2C: Sono aziende che vendono un prodotto che arriva a clienti finali,
ma non hanno il controllo diretto su di essi.
 \item B2G: Business to Government, si ha un'alta complessità legale e
burocratica in quanto si vende direttamente a governi (e quindi lavoro tramite
assegnazione di bandi pubblici e vincite di gare). Anche i metodi di pagamento
sono diversi dalla norma, hanno in media pagamenti più lunghi e che cambiano da
paese a paese.
\end{itemize}

Diverse tipologie di clienti causa un cambiamento nell'approccio che si ha per
la vendita del prodotto.
