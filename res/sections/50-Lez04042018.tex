\chapter{Startup Design Lab}

In questo capitolo si vedrà come portare alla luce le idee per una startup,
partendo dalla genesi, per arrivare al design di un business model efficace per
arrivare alla presentazione del progetto.

\section{Passi per creare una startup}

\begin{itemize}
 \item[Tema] È importante scegliere il tema e formare un team
 \item[Analisi] Fare un'analisi di mercato, individuare i competitor e le
tecnologie
 \item[Idee] Dal brainstorming alla generazione dell'idea
\end{itemize}

\todo{Punti mancanti, copiare dalle slide}

I gruppi variano dalle 3 alle 4 persone, massimo 5. Questa quantità garantisce
una buona variabilità permettendo a tutti i membri di esprimersi.

\subsection{Tema}
Per \textbf{tema} della startup si intende l'argomento generico del progetto,
ma non in dettaglio: si arriva a identificare qual è l'argomento che suscita
curiosità e attorno al quale si vuole generare l'idea di startup. In caso non
ci siano idee, è consigliato seguire questi tre passaggi:
\begin{enumerate}
 \item Scelta dell'elemento tecnologico per la startup su cui far leva per la
scalabilità e per gli altri argomenti che sono stati discussi a lezione e nei
capitoli precedenti
 \item Scelta del tema a cui applicare la tecnologia. È possibile anche
prendere una tencologia già esistente ed applicarla ad un altro contesto
tecnologico
 \item A chi applicare l'idea
\end{enumerate}

\subsection{Analisi}
Una volta trovato il tema, si passa all'analisi di quanto ricavato prima.
È anche importante vedere la mortalità delle startup in quel campo: se essa è
troppo alta infatti significa che è molto probabile incorrere in un fallimento,
o è necessario analizzare in precisione qual è stata la loro causa di decesso.

\subsubsection{Competitor}
In generale, è sempre utile analizzare i \textbf{competitor}, vedere se sono
partiti da startup e/o da chi sono finanziati, e verificare se l'ambito in cui
ci si sta per buttare è già affollato o se c'è poca gente. Anche la dimensione
delle aziende conta: poche piccola startup è molto diverso di avere pochi ma
grandi competitor.

\subsubsection{Tecnologie}
Una volta analizzati i comptetotir è anche importante studiare la
\textbf{tecnologia} che si andrà a usare, le sue caratteritische ma anche i
suoi costi.

\subsubsection{Trend di settore}
È importante andare a studiare come il mercato si evolverà negli anni a venire
in quel determinato settore.
Per esempio, aprire una nuova tipologia di distributore è un'idea a corto
termine, in quanto nel futuro le risorse non rinnovabili andranno ad esaurirsi,
mentre investire in mercati dove sono presenti dei fondi può essere una buona
mossa per convincere anche eventuali investitori per invesire nella realtà che
sis sta tentando di creare.

Il grafico di posizionamento permette di capire eventuali competitor nel
mercato (i \textbf{trend}), per sapere eventuali fasce di mercato ``libere''.
È molto difficile, però, scegliere gli assi da cui fare l'analisi. È possibile
infatti avere valori tutti nello stesso asse, in uno stesso quadrate con
correlati tra di loro (la correlazione si nota quando l'insieme dei punti forma
una diagonale). È altresì fondamentale evitare di scegliere gli assi
soggettivi, come la qualità ad esempio, perché viene difficile eseguirne
un'adeguata valutazione. Gli elementi in questo grafico sono quelli che
emergono dalla analisi eseguita precedentemente sui competitor.

\todo{Aggiungere grafico della slide SDL12}

I social network sono in grado di fare un'analisi di mercato molto precisa,
grazie alla profilazione.

Il posizionamento di un brand è deciso dalle strategie che compie e dal mercato
stesso. Quando le aspettative del brand non sono le stesse del mercato
l'azienda esegue quello che è definito come un riposizionamento di mercato,
ovvero si cambia la percezione degli utenti che hanno sull'azienda stessa. La
difficoltà in tutto ciò è cambiare l'immagine che tutti gli utenti hanno, ed è
un'operazione onerosa e lenta.

\chapter{Curriculum e assunzioni}
\todo{Sta cosa è meglio metterla come appendice secondo me}

\section{La richiesta di lavoro}
Le aziende assumono per vari motivi e per prima cosa come esigenza: emerge
ovvero la necessità di una nuova risorsa umana. Quando si cerca un lavoro in
un'azienda che non è a ricerca di nuove figure probabilmente si sta perdendo
solamente tempo. La nature dell'esigenza può essere generate da due
motivazioni: la prima è che si è aperta una nuova posizione che può essere
perché l'azienda sta aumentando di personale in settori a lei noti e sarà
quindi in grado di supportare anche nuovi arrivati nel posto lavorativo
oppure sta aprendo una nuova posizione in posizioni sconosciute: nel secondo
caso è importante prestare attenzione in quanto non ci sarà nessuno che mi
potrà insegnare e guidare in quella posizione. Da qui è più facile diventare
una figura di riferimento per quell'azienda, ma essere l'unica persona a
gestire un determinato ambito significa avere anche un alto grado di
responsabilità. Uno scenario completamente diverso è la ricerca di un individuo
a causa di una sostituzione: l'azienda ha quindi già esperienza in quel
settore, ma bisogna chiedersi come mai c'è stata la separazione col precedente
lavoratore: che cosa è successo? Si possono avere scenari pericolosi come per
esempio aziende che hanno una certa aspettativa da mantenere proveniente da
lavoratori antecedenti.

La \textbf{job description}, ovvero la descrizione precisa di chi si sta
cercando deve essere precisa e descrittiva. Quando si va a fare un colloquio di
lavoro con qualcuno bisogna avere le idee molto chiare con chi si va a che fare.
Ruolo, skill e caratteristiche personali devono essere presenti nella
descrizione della richiesta di lavoro. \todo{Qui ha fatto l'esempio di quello
che è stato assunto per fare una cosa ma ne finisce per fare un'altra per
evidenziare come anche la definizione del posto sia importante}

\section{Il curriculum}
È il primo strumento di selezione: esiste di tipo cartaceo o digitale. Oramai
il curriculum è quasi sempre digitale, e si usa quello cartaceo durante il
colloquio meramente per prenderci appunti.

Il curriculum non può essere identico per tutte le posizioni lavorative. Una
buoan pratica è enfatizzare le esperienze che si hanno negli ambiti specifici
in cui l'azienda sta cercando lavoro. L'attenzione di chi legge il curriculum
va presa, attraverso parole evidenziate e punti importanti ben spiegati. Questo
non significa mettere tutto nel curriculum, in quanto potrebbe distrarre
l'attenzione di chi sta leggendo il curriculum, annegando le informazioni
inserite per chi sta leggendo. Il livello di dettaglio dovrebbe variare in base
al tipo di curriculum che si vuole ottenere.

Il curriculum non è un mero diario di quello che si è fatto nella propria vita.
Ad esempio formazioni professionali distanti da quelle cercate nell'azienda non
sono utili, e è meglio porle in secondo piano, o potrebbero risultare
fuorvianti.

Non mentire nel proprio curriculum è fondamentale: piuttosto è
meglio scrivere sotto un'altra luce le esperienze fatte. Diversamente da altri
paesi, in Italia è molto importante che i curriculum siano veritieri.

Riguardo la foto sul curriculum, ci può stare come no: in ogni caso, trasmette
tantissimo. L'estetica è uno dei primi elementi con cui le gente viene
valutata, grazie al mid brain.

Nei ruoli dove il contatto con il pubblico è importante la foto è fondamentale,
ma non è necessaria in lavori di ``backend'', dove l'aspetto non è necessario,
metterla quindi non è obbligatorio ma potrebbe diventare un'arma a doppio
taglio.

\subsection{Social network}
Riguardo ai social network, bisogna prestare un po' di attenzione. LinkedIn va
tenuto sempre in ordine, mentre su Facebook bisogna guardare i contenuti
presenti.

\subsection{Lettera di presentazione}
Una lettera di presentazione è una lettera che introduce una persona e ne
esalta le caratteristiche per quel determinato posto di lavoro. Rappresenta
tutte le caratteristiche emotive ed emozionali che non sono inseribili nel
curriculum, ed in certi casi è altrettanto importante del curriculum, e
può essere di accompagnamento. Anche questa, non dovrebbe essere ne asettica ne
copia-incollata.
In una lettera di presentazione bisogna far capire quanto quell'azienda è
importante per il candidato, non il contrario.
