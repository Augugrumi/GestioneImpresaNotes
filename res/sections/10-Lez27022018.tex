\chapter*{Introduzione sul corso}

Il corso varia in base all'interesse degli studenti stessi, quindi i contenuti
possono cambiare di anno in anno.

\paragraph*{Libri di testo} I libri di testo adottati sono:
\begin{itemize}
 \item Startup Marketing: errori da evitare e strategie da seguire (A.
Baldissera, B. Bonaventura)
 \item Business Model Generation: a handbook for visionaries, game changers,
and challengers (A. Osterwalder, Y. Pigneur)
\end{itemize}

\paragraph*{Esame} Ci sono due modalità differenti per frequentanti e non: 
\begin{itemize}
  \item \textbf{Frequentanti}: prima della fine della lezione ci sarà un
  compitino sugli argomenti trattati a lezione, a cui segue un progetto da
  svolgere in gruppi (ideazione e creazione di una start-up). Alla fine del
  corso ci sarà la presentazione di tale progetto e un eventuale orale
  opzionale per migliorare il voto o per mirare alla lode. Chi frequenta avrà
  due punti bonus, consegnando il progetto alla prima sessione d'esame. I
  gruppi vanno da quattro a sei persone.

  \item \textbf{Non frequentanti}: chi non frequenta dovrà sostenere sia
  l'esame scritto che orale (obbligatorio).
\end{itemize}
L'esame scritto è composto da un gruppo di domande chiuse (a crocette) seguite
da un'insieme di domande aperte. Non vengono rimossi punti se la risposta data è
sbagliata (né per le domande aperte, né per le domande a risposta multipla).

\paragraph*{Ricevimento} Il professore non fa ricevimento, ma è possibile
contattarlo sia per email che su Facebook (in privato o nel gruppo chiuso di
Gestione di Imprese).


\chapter{Introduzione}

Quando si parla del mondo del lavoro spesso si sente parlare di molti termini,
parole. Cominciamo a capire di che cosa stiamo parlando: che cos'è un
imprenditore secondo voi? Si potrebbe parlare di qualcuno che ha i soldi (anche
se non è così), ma anche di qualcuno che da vita a un'attività, ovvero
trasforma dei capitali (investimento iniziale) in un motore di guadagno
economico.

Cos'è invece un manager invece? Un manager è qualcuno che gestisce un'attività
che è già avviata.

Attività, società, impresa, azienda sono tutti sinonimi, ma cosa significano?
Un'attività un'insieme di persone il cui scopo è offrire un servizio/prodotto
dietro guadagno.

\paragraph*{Società e liberi professionisti} Società e liberi professionisti a
livello concettuale possono essere visti come la stessa cosa, ovvero come
entità giuridiche, riconosciute dallo Stato, produttrici o erogatrici di
beni/servizi per i quali qualcuno è disposto a pagare. Da ciò generando flussi
di denaro, identificati dallo stato (grazie ad esempio la partita IVA), per i
quali possono essere tassati (è per questo che è necessario il contenitore
giuridico).

Un libero professionista è una persona (o un piccolo gruppo di persone che
collaborano es. un elettricista e i suoi apprendisti) che possiede delle
conoscenze in un certo campo e le sfrutta nel mondo del lavoro dietro compenso.
Le società, invece, non si basano sulle competenze di una singola persona, ma su
di una organizzazione.

Le società, infatti, sono delle strutture più organizzate, che
impiegano molteplici individui per produrre un prodotto o offrire un servizio
tendenzialmente complessi, per cui il singolo individuo non è
sufficiente (e per questo vengono, solitamente, tassate di più).

I liberi professionisti possono:
\begin{itemize}
  \item avere contratti differenti, ad esempio a chiamata;
  \item essere di vario tipo ad esempio ingegneri, Architetti, Idraulico,
  etc\dots ovvero tutti coloro che hanno delle competenze specifiche
\end{itemize}
Rispetto ad una società che necessità di un gruppo di persone per produrre o
erogare un servizio, il libero professionista è lui stesso l'erogatore del
servizio.

\paragraph*{Imprenditore} Un'imprenditore è un individuo che ha la
proprietà di un'azienda/società. Nelle aziende unipersonali, c'è un individuo
che ha il 100\% della proprietà dell'azienda, ovvero possiede l'azienda perché
è un qualcosa che ha pagato. La società è una struttura giuridica separata
rispetto l'imprenditore, di conseguenza i capitali di un'azienda non sono di
fatto di chi l'ha fondata o di chi la possiede (che magari ne possiedono il
controllo indiretto), ma dell'azienda stessa. Un'imprenditore, per esempio, per
riottenere i soldi investiti dovrebbe vendere una parte di quella struttura per
capitalizzare. Nel caso di un \textit{libero professionista} invece è
differente: persona e società sono la stessa cosa e quindi il professionista
detiene il capitale. L'imprenditore è sempre e necessariamente proprietario
della società indipendentemente dal suo ruolo all'interno della stessa.
La struttura unipersonale può funzionare solamente nelle piccole imprese,
mentre quando le strutture si evolvono, le società solitamente non appartengono
più ad una singola persona, ma ad un un gruppo di imprenditori o di altre
aziende. A questo punto ogni singolo/azienda ha una percentuale di questa
società. Tutti coloro che ne possiedono una parte della società vengono
definiti come \textbf{consoci}, ovvero si ha una partecipazione all'interno di
quella società, indipendentemente dal fatto che il socio ci lavori o meno
(esistono infatti imprenditori con figure attive o passive). Infine, è
importante ricordare che l'imprenditore ha una responsabilità sull'azienda.

\paragraph*{Manager} A differenza dell'imprenditore che possiede un'azienda ma
potrebbe non lavorarci all'interno, il \textbf{manager} è una persona che non ha
la proprietà dell'azienda, ma ne è un dipendente e la gestisce.
Ad esempio, la FIAT è degli Agnelli, che sono gli imprenditori (e non
dipendenti della società), ma viene gestita da Marchionne, che è un manager e,
quindi, un dipendente. Il manager potrebbe avere più potere decisionale di 
alcuni imprenditori: tutto dipende dagli accordi che sono stati siglati.

\subparagraph*{Differenze tra Manager e Imprenditore}
\begin{itemize}
  \item l'imprenditore non è un dipendente dell'azienda;
  \item l'imprenditore, a parte i soci/altri imprenditori/azionisti, non deve
rispondere a nessun altro;
  \item l'imprenditore non ha un contratto da dipendente (in generale; se 
  lavora in azienda ha un contratto);
  \item l'imprenditore non deve avere competenze particolari per quanto riguarda
l'azienda (questo è vero solo in caso di aziende grandi)
  \item l'imprenditore vive dei ricavi della società/impresa (ne prende una
percentuale), mentre il manager è un dipendente, ha un contratto di lavoro che
ne determina uno stipendi che non dipende necessariamente dai ricavi
dell'azienda\\[0.5cm] \todo{il prof ha detto che è un concetto ``estremamente
interessante''}
\end{itemize}

\subparagraph*{Fatturato e Utile} Per fatturato ci si riferisce ai soldi che
l'azienda genera tramite le vendite o/e l'erogazione di servizi. Se dal
fatturato togliamo le tasse e le varie spese di produzione otteniamo l'utile
dell'azienda: quest'ultimo può essere re-investito nell'azienda o diviso tra
i soci. In questo caso viene definito \textit{dividendo} quanto ogni socio
guadagna. Sono possibili, ovviamente, anche soluzioni intermedie in cui una
parte degli utili viene re-investita e un'altra viene distribuita tra i soci.
