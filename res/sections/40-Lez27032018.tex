\chapter{Neurobusiness}

La Neurobusiness è nata a seguito di un paio di evoluzioni pesanti che sono
avvenute nell'ultima decina danni.

Il concetto dietro a ciò è che il cervello umano sua uno strumento di alta
qualità, prodotto da 4 miliardi di anni di evoluzione. Uno dei frutti
dell'evoluzione sul nostro pianeta più importante, in quanto ci ha portato ad
essere una delle specie dominanti del nostro pianeta.

Questo è di conseguenza uno strumento enormemente potente. Dall'inizio della
cultura classica (degli antichi greci) la mente umana è ciò che distingue gli
umani dagli animali. Il pensiero razionale è opposto a quello emotivo,
istintivo, dettato dalle emozioni. Gli istinti sono più basici, inferiori, e
solo quello razionale può essere definito come inteso come un \textit{pensiero}.
Ma questo sbilanciamento è giusto? È vero che siamo sempre così razionale?

\paragraph*{Il pensiero razionale} Il pensiero razionale è:
\begin{itemize}
 \item Consapevole: l'individuo ha delle preferenze ed è in grado di conoscerle
(preferisce il rosso al verde, le pere alle mele, e la destra alla sinistra).
Ossia sa paragonare due cose anche, eventualmente, per trovarle indifferenti
 \item Consistente: le preferenze dell'individuo sono \textbf{transitive} (se $A
> B$ e $B > C$ allora $A > C$, ossia, se preferisco le pere alle mele e le mele
le trovo più buone delle banane, prefeririò le pere alle banane)
 \item Finalizzato: l'individuo \textbf{sa cosa vuole} e si comporta per
cercare di ottenerlo
\end{itemize}

Questo tipo di pensiero è meccanicistico, basato su questi pilastri.
L'\textit{homo economicus}, ovvero l'individuo standard su cui sono stati
individuati i principi dell'economia meccanicistica, si basa sulla legge della
domanda e dell'offerta.
Essendo l'individuo molto differente è stato generalizzato, basandolo sui
pilastri sopra elencati e sull'egoismo assoluto: se cerco di vendere un
prodotto cerco di venderlo al prezzo più alto possibile, se cerco di
acquistarlo su quello più basso invece.

C'è un problema: per gli esseri umani è un problema fare i conti a grandi
distanze nel tempo: vale di più la gratificazione immediata che quella a lungo
termine.

\paragraph*{Irrazionalità} Vengono considerate tre aree del cervello: quello
antico (\textit{old brain}), detto anche cervello rettile, ovvero il primo a
evolversi/manifestarsi nel profilo filogenetico ed è colui che si attacca alla
spina dorsale e contiene pure funzione di sopravvivenza. La seconda parte, il
\textit{mid brain} contiene le emozioni: emozioni positive e sensazioni
negative. L'ultima parte (anche l'ultimo ad essersi evoluto), è chiamata
\textit{new bain} ed è dedita al ragionamento consapevole: questa parte è
ulteriolmente suddivisa in parti diverse (frontale, occipitale, ma anche destra
e/o sinistra).

Noi siamo convinti che le nostre decisioni siano basate su elementi razionali,
ma invece si prendono decisioni non sulla base di informazioni concrete ma su
processi inconsapevoli e istintivi.
È un male? No, è grazie proprio a questi elementi che la nostra specie si è
evoluta ed è diventata quella che è oggi: il problema principale è che tutti i
nostri sensi mandano continuamente informazioni al nostro cervello, che
forniscono all'incirca 11 milioni di informazioni al secondo, di cui solo una
piccolissima parte, 40,  è in grado di essere percepita e usata dalla
razionalità. La mente razionale è quindi uno strumento potentissimo ma è un
modo di pensare lento. Per sopravvivere, gli umani e l'essere umano hanno
bisogno, spesso,\todo{Non ha mai finito questa frase!}

\section{L'economia e la psicologia}

Ad un certo punto la psicologia e l'economia si sono incontrate, da una parte
avendo le fRMI (risonanza magnetica funzionale)\footnote{È una risonanza
magnetica in cui si prende dello zucchero a cui si aggiunge un liquido di
contrasto. Questo va bevuto, ed essendo il cervello la parte del corpo che
consuma più zuccheri tenderà ad essere maggiormente assorbito dal cervello. Le
aree più attivate dal cervello avranno maggior bisogno di zuccheri, dove ci
sarà quindi una maggiore assunzione dello stesso.}. La fRMI permette di
eseguire analisi in tempo reale, permettendo di denotare il comportamento del
cervello.

Dal punto di vista economico ci si è scontrati con il fallimento dei sistemi
previsionali classici: crisi dei sistemi finanziari per esempio, dove i sistemi
di economia meccanicistica non è stato in grado di prevedere cosa stesse
accadendo.

Da questi due punti c'è stata una convergenza, che ha portato al
\textbf{Neurobusiness}.

\subsection{Il metro di paragone}

Irrazionale significa che la decisione che viene presa non si basa su un flusso
razionale che avremmo di solito, ma esiste un elemento particolare su cui
è possibile creare delle teorie? Si, e questo è che è la sua prevedibilità.

Noi, come esseri umani, non siamo in grado di misurare le cose in termini di
valore assoluto.
\todo{Aggiungere l'esempio dell'abbonamento al newpaper che me lo sono perso
perché lo stavo ascoltando :D}
Le scelte separate del contesto risultano essere difficili da fare: le scelte
dipendono sempre dal contesto, portandoci a fare scelte in termini relativi e
non assoluti.

\todo{Manca l'esempio della penna e della giacca}

\subsection{Ciò che è mio è mio}
Quando si possiede qualcosa si tende a valutarla di più di quello che farebbero
gli altri.

\todo{Manca l'esempio delle partite di basket}

Gli oggetti che possediamo vengono caricate di ricordi, e così facendo ne
modifichiamo il valore percepito. Quando vendiamo qualcosa ci concentriamo di
più su cosa perderemmo che su cosa guadagneremmo (la \textit{loss aversion} ci
evidenzia più i costi dei benefici), e comunque partiamo dal presupposto che
gli altri vedano la transazione dal nostro stesso punto di vista, quindi con i
nostri valori e i nostri ricordi.
Questo è dovuto alla nostra natura, ovvero che tenta di evitare i rischi: ciò
si traduce anche quando cambiamo marca di un prodotto.
È molto più pesante il timore della perdita rispetto alla percezione di
guadagno, con un rapporto di 4 a 1.

\subsection{Ciò che potrebbe essere mio è mio}
La proprietà virtuale, ossia la tendenza ad avere la sensazione di possedere un
oggetto prima ancora di esserne i reali proprietari, ne aumenta il valore. Ad
esempio, durante le aste online, maggiore è il tempo che un individuo passa come
miglio offerente (anche in assenza di altre offerte) più egli si sentirà di
possedere già l'offetto e più alto diventa il rialzo massimo che sarà disposto a
fare.

\subsection{Come sfruttare ciò da un punto di vista del business}
È su questo principio che si basano i periodi di prova dei prodotti e le prove
gratuite delle autovetture: una volta acquistato il pacchetto base se viene
offerto il pacchetto premium per un periodo limitato per il cliente diventerà
molto difficile disdirlo.
