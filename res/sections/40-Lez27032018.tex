\chapter{Neurobusiness}

Il Neurobusiness è nato a seguito di un paio di evoluzioni pesanti che sono
avvenute nell'ultima decina d'anni.

Precedentemente la considerazione che si aveva del cervello era che fosse uno
strumento di alta qualità, prodotto da 4 miliardi di anni di evoluzione, frutto
dell'evoluzione e poiché ci ha portato ad essere una delle specie dominanti del
nostro pianeta allora doveva per forza essere perfettamente razionale.

Dall'inizio della cultura classica (degli antichi greci) la mente umana è
considerata ciò che distingue gli umani dagli animali. Il pensiero razionale è
opposto a quello emotivo, istintivo, dettato dalle emozioni. Gli istinti sono
più basici, inferiori, e solo quello razionale può essere definito un
\textit{pensiero}. Ma questo sbilanciamento è giusto? È vero che siamo
sempre così razionali?

\paragraph*{Il pensiero razionale} Il pensiero razionale è basato su questi
assiomi:
\begin{itemize}
 \item \textbf{Consapevolezza}: l'individuo ha delle preferenze ed è in grado
 di conoscerle (preferisce il rosso al verde, le pere alle mele, e la destra
 alla sinistra). Ossia sa paragonare due cose anche, eventualmente, per
 trovarle indifferenti;
 \item \textbf{Consistenza}: le preferenze dell'individuo sono \emph{transitive}
 (se $A > B$ e $B > C$ allora $A > C$, ossia, se preferisco le pere alle mele e
 le mele le trovo più buone delle banane, preferirò le pere alle banane);
 \item \textbf{Finalizzato}: l'individuo \emph{sa cosa vuole} e si comporta in
 maniera tale per cercare di ottenerlo.
\end{itemize}
Da questa visione dell'uomo è nata l'\textbf{economia meccanicista}, basata
sulla legge della domanda e dell'offerta. La persona, per astrarre dalle
differenze di ogni singolo individuo, è generalizzata dall'\textit{homo
economicus}. Esso si basa sui pilastri sopra elencati e, di conseguenza,
sull'egoismo assoluto: se cerco di vendere un prodotto cerco di venderlo al
prezzo più alto possibile, se cerco di acquistarlo su quello più basso invece.

C'è un problema: gli esseri umani non sono così come si pensava e per esempio
non è in grado di fare i conti a grandi distanze temporali: il valore di una
gratificazione immediata è considerato molto maggiore rispetto ad una
ricompensa a lungo termine.

\paragraph*{Irrazionalità} 
Il cervello umano può essere suddiviso in tre macro-aree: quello
antico (\textit{old brain}), detto anche cervello rettile, ovvero il primo a
evolversi/manifestarsi nel profilo filogenetico. Ad esso è attaccata alla
spina dorsale e si occupa di pure funzioni di sopravvivenza.
La seconda parte, il \textit{mid brain}, si occupa delle emozioni: emozioni
positive, delle sensazioni negative e dei ricordi.
L'ultima parte, che corrisponde all'ultima area che si è sviluppata, è chiamata
\textit{new brain} ed è dedita al ragionamento consapevole: questa parte è
ulteriormente suddivisa in diverse parti (frontale, occipitale, ma anche
emisfero destro e sinistro).

Noi siamo convinti che le nostre decisioni siano basate su elementi razionali,
ma spesso, invece, si prendono decisioni non sulla base di informazioni
concrete ma su processi inconsapevoli e istintivi.
È un male? No, è grazie proprio a questi elementi che la nostra specie si è
evoluta ed è diventata quella che è oggi: il problema principale è che tutti i
nostri sensi mandano continuamente informazioni al nostro cervello, che
forniscono all'incirca 11 milioni di frammenti di informazioni al secondo, di
cui solo una piccolissima parte (40!) è in grado di essere percepita ed usata
razionalmente. La mente razionale è quindi uno strumento potentissimo, ma anche
molto lento. Per sopravvivere, gli umani hanno bisogno, spesso, di reazioni più
veloci: per gli atleti spesso è indispensabile la memoria muscolare.

\section{L'economia e la psicologia}

Ad un certo punto la psicologia e l'economia si sono incontrate.
Grazie alla fRMI (risonanza magnetica funzionale)\footnote{È una
risonanza magnetica in cui si prende dello zucchero a cui si aggiunge un
liquido di contrasto. Questo va bevuto, ed essendo il cervello la parte del
corpo che consuma più zuccheri tenderà ad essere maggiormente assorbito dal
cervello. Le aree più attivate dal cervello avranno maggior bisogno di
zuccheri, dove ci sarà quindi una maggiore assunzione dello stesso.} è stato
possibile analizzare in tempo reale il cervello, permettendo di capire quali
sono le parti coinvolte in determinati processi e azioni.

Dal punto di vista economico ci si è scontrati con il fallimento dei sistemi
previsionali classici: la crisi globale dei sistemi finanziari iniziata nel
2007 ne è un esempio: i sistemi di economia meccanicistica non sono stati in
grado di prevedere cosa stesse accadendo.

Dalla convergenza di questi due mondi è nato il \textbf{Neurobusiness}.

\subsection{Il metro di paragone}

``Irrazionale'' significa che la decisione che viene presa non si basa su un
flusso razionale, che avremmo di solito. Esiste un elemento particolare su cui
è possibile creare delle teorie? Sì, ovvero la prevedibilità del pensiero
irrazionale.

Noi, come esseri umani, non siamo in grado di misurare le cose in termini di
valore assoluto.
Un esempio può essere il seguente. The Economist, qualche tempo fa, ha 
proposto 3 tipi di abbonamenti:
\begin{enumerate}
 \item Solo online per 59\textdollar{};
 \item Solo cartaceo per 125\textdollar{};
 \item Online + cartaceo per 125\textdollar{}.
\end{enumerate}
In questo caso il 16\% degli abbonati scelse la prima opzione e l'84\% la terza.
Togliendo la seconda opzione invece le percentuali si sono stravolte, con il
68\% degli abbonati che hanno scelto il cartaceo e 32\% l'abbonamento per
l'online e il cartaceo. 
Questo fenomeno è dovuto al fatto che un individuo non è in grado di scegliere
tra 2 elementi completamente differenti, mentre se c'è una terza opzione che fa
risaltare una delle due precedenti, palesemente sub-ottimale, allora sceglierà
nella maggior parte dei casi la soluzione enfatizzata.
\todo{sistemare percentuali}

Per noi umani risulta difficile fare scelte separate dal contesto: è più
facile basarsi su di un contesto per prendere una decisione poiché siamo più 
propensi a ragionare in termini relativi e non assoluti.

\begin{example}[Penna e Vestito]
Un esempio è il seguente. Supponiamo di essere in un negozio nel quale
la penna che vogliamo costi 25\textdollar{} e che un altro acquirente ci dica
che ad un negozio a 15 minuti a piedi di distanza la stessa identica penna
costi 18\textdollar{}. In questo caso la maggioranza cambierà negozio. Tale
risultato verrà ribaltato invece se parliamo di una giacca che costa
450\textdollar{} in un negozio e 443\textdollar{} nell'altro anche se, di
fatto, il risparmio è sempre di 7\textdollar{}.
\end{example}

\subsection{Ciò che è mio è mio}
Quando si possiede qualcosa si tende a darle un valore più alto rispetto a
quello dato da un'altra persona.
Uno studio in questo campo è stato fatto sul valore associato ai biglietti per
le partite di basket della squadra dell'università del Duke University
\todo{sistemare nome università},
nella quale il basket è quasi una religione. Per
cercare di comprare un biglietto si è sottoposti a sorteggi casuali e levatacce
durante la notte. È stato chiesto quindi a chi ha provato a comprare un
biglietto e non è riuscito e chi c'è riuscito il valore che gli attribuivano.
In media i primi hanno risposto 175\textdollar{}, i secondi 2400\textdollar{}!
Questo accade perché gli oggetti che possediamo vengono caricati di ricordi, e
così facendo, ne modifichiamo il valore percepito. Quando vendiamo qualcosa ci
concentriamo di più su cosa perderemmo che su cosa guadagneremmo (la
\textit{loss aversion} ci evidenzia più i costi dei benefici), e comunque
partiamo dal presupposto che gli altri vedano la transazione dal nostro stesso
punto di vista, quindi con i nostri valori e i nostri ricordi.
Questo è dovuto alla nostra natura, che tenta di evitare i rischi: ciò
si traduce anche quando cambiamo marca di un prodotto.
È molto più pesante il timore della perdita rispetto alla percezione di
guadagno, con un rapporto di 4 a 1.

\subsection{Ciò che potrebbe essere mio è mio}
Anche proprietà virtuale, ossia la tendenza ad avere la sensazione di possedere
un oggetto prima ancora di esserne i reali proprietari, aumenta il valore di un
bene per una persona. Ad esempio, durante le aste online, maggiore è il tempo
che un individuo passa come miglior offerente (anche in assenza di altre
offerte) più egli si sentirà già in possesso dell'oggetto e più alto diventa il
rialzo massimo che sarà disposto a fare.

\subsection{Come sfruttare ciò da un punto di vista del business}
È su questo principio che si basano i periodi di prova dei prodotti e le prove
gratuite delle autovetture: una volta acquistato il pacchetto base se viene
offerto il pacchetto premium per un periodo limitato per il cliente diventerà
molto difficile disdirlo. In modo analogo, durante le prove delle auto vengono
mostrate sempre i ``full optional'', rendendo più difficile al possibile
acquirente rinunciarci.
