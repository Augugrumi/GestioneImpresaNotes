\subsection{La forza dell'aspettativa}

L'aspettativa è l'interpretazione che io do prima che quel qualcosa accada.
Tali aspettative influenzano la percezione della realtà di un individuo.

In un esperimento si son prese due caraffe di birra, contenenti 2 birre
differenti: una normale e una corretta con alcune gocce di aceto balsamico.
Ai soggetti (i normali avventori di un pub) veniva dato un assaggio di entrambe
poi un bicchiere della birra che preferivano.
I volontari dell'esperimento sono stati divisi in due gruppi: al primo gruppo
non venne spiegata la differenza tra le birre, mentre al secondo sì. Il
risultati dicono che il primo gruppo scelse la birra con l'aceto, il secondo la
birra normale. Ciò è dovuto all'aspettativa: oggettivamente la birra con
l'aceto balsamico è buona, ma la nostra aspettativa ci induce a pensare che non
lo sia.

L'esperimento è poi proseguito aggiungendo un terzo gruppo, al quale la
differenza tra le 2 caraffe di birra è stata spiegata dopo l'assaggio, ma
prima della scelta e, anche in questo caso, la maggioranza scelse la birra
piacque con l'aceto.

Tale esperimento dimostra quindi che l'aspettativa aveva modificato
effettivamente la percezione del soggetto, lo stimolo gustativo stesso.

L'aspettativa è una forma di pensiero definita di alto livello: quel qualcosa
viene interpretato qui e ora, e proiettata nel futuro. Questi livelli di
elaborazione nel cervello sono molto alti, e sono in grado di influenzare
elementi molto più bassi di livello di informazione, tanto da arrivare a
modificare elementi di percezione. Quindi se io creo delle aspettative su
qualcosa, in qualsiasi modo questa viene fatta, si avranno della sensazioni
diverse. Anche le etichette del cibo lavorano in questa maniera: più creano una
buona aspettativa, più effettivamente una persona troverà tale pietanza buona!

Quando una aspettativa viene delusa si hanno delle grosse ripercussioni, anche
se, in generale, creare una buona aspettativa fa sì che un prodotto venga
considerato un maniera più positiva.

Un altro esempio in cui l'aspettativa gioca un ruolo chiave è l'impiattamento.
Tutti noi preferiamo mangiare un piatto ben curato e questo perché altera la
nostra aspettativa: migliore sarà l'impiattamento, maggiore sarà la nostra
aspettativa, più la pietanza ci sembrerà buona.

Un ultimo esempio è il \emph{packaging} di un prodotto: non si può insinuare
che la confezione valga tanto come il prodotto al suo interno ma contribuisce
alla soddisfazione dell'acquirente. Prodotti come i profumi spesso presentano
boccette molto elaborate proprio per questo scopo.

Tutto ciò ci insegna che il primo impatto e l'apparenza sono molto importanti,
e permette di influenzare la qualità del prodotto percepita dall'utente. Non
tutti ne sono influenzati allo stesso modo, ma nessuno ne è completamente
indifferente.

\subsection{La forza del brand}

Quanto arriva a essere potente la forza di un brand, non solo da un punto di
vista economico ma anche dal punto di vista dell'aspettativa?
Il brand, uno degli asseti più significativi per un'azienda si basa
pesantemente sull'aspettativa del cliente nei confronti di quella marca.

In un test d'esempio tra la Coca-cola e la Pepsi è stato utilizzato l'fRMI. In
un test cieco si attivano solo le aree deputate alla percezione gustativa e
molti soggetti finivano col preferire la Pepsi. Se i soggetti vengono
informati sulla bibita che stanno per assaggiare, si attivano anche altre
aree cerebrali, totalmente slegate al gusto, ma legate alle emozioni e ai
ricordi (come l'ippocampo e la corteccia prefrontale dorso laterale, facenti
parte del mid brain). Queste venivano attivate in quantità maggiore nel caso
della Coca-cola e, difatti, la maggior parte dei soggetti, in questo caso,
preferì quest'ultima. Sono questi processi cognitivi aggiuntivi che danno un
vantaggio di mercato alla Coca Cola, non le sue proprietà chimiche.

Quando il brand è noto e si hanno alte aspettative si attivano tutta una serie
di processi cognitivi, che altrimenti non verrebbero attivati. Questo è la vera
potenza di un brand: se un marchio è in grado di attivare il nostro ippocampo
ed evocare emozioni e ricordi in noi vuol dire che può facilmente influenzare
le nostre decisioni.

Queste considerazioni non sono limitate agli oggetti e ai marchi, ma anche alle
persone. Anche per questo, negli ultimi anni, è nato il \textbf{self-branding}, ovvero la gestione di se\' stessi e della propria reputazione come fosse un brand. A questo sono strettamente legate tutte le attività di \textbf{self-promoting}.\todo{Parte della definizione manca.} Queste attività sono particolarmente importanti per consulenti, youtuber, politici\dots{}

\subsection{La reputazione}

Nel passato, gli acquisti di determinati prodotti costruiva attorno a un brand
una determinata reputazione. Con l'avvento della pubblicità di
massa\footnote{Detta anche pubblicità \textit{broadcast}. Nota: la sua
efficacia cala di anno in anno, in quanto l'utente è più informato rispetto al
passato, grazie soprattutto ad Internet.}, e con la popolazione non
adeguatamente informata, i brand potevano facilmente pubblicizzare
il loro prodotto come il migliore. Soprattutto grazie al Web 2.0 ora
è più facile per i consumatori diffondere la loro opinione ad altri su un
determinato prodotto, rendendo la vita più ``difficile'' ai brand. In questa
maniera i brand hanno bisogno di mantenere un'alta reputazione e devono
prestare attenzione alla comunicazione con gli utenti.

La \textbf{brand reputation} è la gestione della reputazione di un marchio, che
comprende anche il comportamento associato alle persone dell'azienda e la sua
presenza sui social.

\subsection{Gamification}

Con \emph{gamification} si intende l'utilizzo di tecniche, regole e dinamiche
tipiche del mondo dei giochi in contesti diversi da quelli ludici. Significa
utilizzare meccaniche e dinamiche di gioco come punti, livelli, reward,
missioni e status all'interno di contesti non gaming per creare engagement e
risolvere problemi.
Uno strumento in grado di agire visceralmente \todo{Ha cambiato slide :(}

Esistono diversi universi:
\begin{itemize}
 \item dei giochi
 \item del divertimento (per esempio tra amici)
 \item i social game
 \item lavoro
 \item vita
\end{itemize}
\todo{Mettere immagine al posto dell'elenco}

La gamification significa unire il gioco e divertimento con il resto degli
universi. Attenzione! Non bisogna confondere \emph{gamification} e \emph{game
design}. La \emph{gamification} infatti aggiunge meccaniche gaming su un
prodotto pensato per finalità differenti, mentre il \textit{game design} si
occupa di dar vita ad un gioco vero e proprio, un prodotto profondo ed
immersivo che non risulti né troppo difficile né troppo complicato da imparare
in modo di far si che l'utente utilizzi il gioco per il maggior tempo possibile
senza annoiarsi.

In cosa può essere usata la gamification? Questa si differenzia dalla
pubblicità (pubblicizzo un mio prodotto) e dalla promozione (ad es. prendi 3
paghi 2, sorpresa negli oggetti che si comprano). Nella pubblicità, infatti, si
incentiva la brand awareness, ha bisogno di grossi investimenti e si hanno
guadagni a lungo termine: anche nel caso in cui si sospenda un determinato spot
pubblicitario i suoi effetti rimarranno per un determinato periodo di tempo. Le
promozioni, invece, non aumentano la brand awareness, ma incentivano
immediatamente le vendite per la durata della promozione stessa. La
gamification \textbf{non} è una promozione, e coinvolge un sistema a
punteggio/livelli, a cui sono collegate solitamente delle reward e può avere un
certo grado di competizione.
La gamification è efficace perché è una tecnica forte per scardinare le
abitudini. Le abitudini sono dei comportamenti radicati nell'individuo,
difficili da modificare (ad esempio cambiare un'abitudine alimentare è
faticoso) poiché ci sarebbe un costo fisico o emotivo da pagare, anche se
razionalmente si sa che modificare una determinata abitudine potrebbe portare a
dei benefici. Questa tecnica funziona bene sopratutto nell'ambito delle diete e
del fitness. La capacità di questa tecnica di rompere le regole aggiunge una
variabile nella scelta compiuta da nostro cervello che non è più solamente tra
una gratificazione immediata (gestita dal mid brain) o a lungo termine 
ragionata e quindi che lavora sul new brain)m, che può essere per esempio la
gratificazione sociale (\emph{like} ai post, incitamento mentre si corre da
parte di altri utenti, \dots{}).

È importante nella vendita dei prodotti tenere conto di come funziona il nostro
cervello, quindi:
\begin{itemize}
 \item alcuni prodotti vengono scelti e valutati dalla nostra mente razionale
 (ad esempio le tariffe telefoniche);
 \item altri parlano con le nostre sensazioni e/o ricordi (trailer del cinema,
 cibo).
\end{itemize}

Le pubblicità devono quindi parlare con le aree preposte del nostro cervello,
cioè nel primo caso al new brain, nel secondo al mid brain. È importante
\textbf{non mescolare} i diversi stili di comunicazione, in quanto prevale
sempre la parte emozionale su quella razionale, in quanto gli aspetti del mid
brain sono più legati alla sopravvivenza o alla riproduzione. Non ha senso
quindi sponsorizzare delle tariffe telefoniche con un/una bel/la ragazzo/a
mezzo nudo/a poiché si deve parlare con la parte razionale e non emozionale.

Per riassumere, la gamification inserisce degli elementi di dialogo
con la mente emozionale.
