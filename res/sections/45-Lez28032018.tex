\subsection{La forza dell'aspettativa}

L'aspettativa è l'interpretazione che io do prima che quel qualcosa accada.

In un esperimento si son prese due caraffe di birra. Ce n'erano di due tipi:
una normale e una corretta con alcune gocce di aceto balsamico. Ai soggetti (i
normali avventori di un pub) veniva dato un assaggio di entrambe poi un
bicchiere\todo{Ha skippato le slide non ho fatto in tempo a copiare D:}.
Al primo gruppo non venne detto cosa differenziava le due birre e la maggior
parte scelse quella adulterata. Al secondo gruppo venne detto dell'aceto e
quasi tutti scelsero quella normale.
Questo è dovuto all'aspettativa: oggettivamente la birra con l'aceto balsamico
è buona, ma la nostra aspettativa ci induce a dire che non lo è.

Quando, in un terzo caso, la notizia dell'aceto venne data dopo l'assaggio, ma
prima della valutazione, la birra piacque.
Quindi l'aspettativa aveva modificato effettivamente la percezione del
soggetto, non solo il suo giudizio informato.

L'aspettativa è una forma di pensiero definita di alto livello: quel qualcosa
viene interpretato qui e ora, e proiettata nel futuro. Questi livelli di
elaborazione nel cervello sono molto alti, e sono in grado di influenzare
elementi molto più bassi di livello di informazione, tanto da arrivare a
modificare elementi di percezione. Quindi se io creo delle aspettative su
qualcosa, in qualsiasi modo questa viene fatta, si avranno della sensazioni
diverse.

Quando una aspettativa viene delusa si hanno delle grosse ripercussioni, anche
se è vero che avere una aspettativa alta migliora la
percezione\todo{ricontrollare questo punto}.

\todo{Esempio dell'impiattamento}

\todo{Esempio del packaging di un prodotto su un prodotto stesso}

Tutto ciò ci insegna che il primo impatto e l'apparenza sono molto importanti,
e permette di influenzare la qualità del contenuto percepita dall'utente.

\subsection{La forza del brand}

Quanto arriva a essere potente la forza di un brand, non solo da un punto di
vista economico ma anche dal punto di vista dell'aspettativa?
Il brand, uno degli asseti più significativi per un'azienda si basa
pesantemente sull'aspettativa del cliente nei confronti di quella marca.

In un test d'esempio tra la Coca-cola e la Pepsi è stato applicato l'fRMI. In
un test cieco si attivano solo le aree deputate alla percezione gustativa e
molti soggetti finivano col preferire la Pepsi. Se i soggetti vengono
informati, e sanno quale bibita stanno per assaggiare, si attivano anche altre
aree cerebrali più legate alle sensazioni emotive (l'ippocampo e la corteccia
prefrontale dorso laterale) e queste si arrivano in misura maggiore per la
Coca-cola )e la maggior parte dei soggetti preferì la Coca). Sono questi
processi cognignitivi aggiuntivi che danno un vantaggio di mercato alla Coca
Cola, non le sue proprietà chimiche.

Quindi quando il brand è noto e ha alte aspettative si attivano tutta una serie
di processi cognitivi che non si attiverebbero altrimenti. Questo è il
significato della potenza del brand. L'ippocampo in particolare è interessato
in quanto è la zona adibita ai ricordi, e associa il prodotto a i ricordi
dell'utente finale. Riuscire a far si che la parte dell'ippocampo si attivi è
molto importante.

Self-branding: gestione di se stessi come se si fosse un brand, a cui sono
legate tutte le attività di self-promoting.\todo{Parte della definizione manca.}

\subsection{La reputazione}

Nel passato, gli acquisti di determinati prodotti costruiva attorno a un brand
una determinata reputazione. Con l'avvento della pubblicità di
massa\footnote{Detta anche pubblicità \textit{broadcast}. Degna di nota,
l'efficacia della pubblicità \textit{broadcast} cala di anno in anno, in
quanto è più informato rispetto al passato.}, e con la popolazione non
adeguatamente informata, i brand potevano facilmente diffondere il loro
prodotto come il migliore. Con l'avvento della società di massa e della
globalizzazione è stato più facile per i consumatori diffendere la loro opinione
ad altri su un determinato prodotto, rendendo la vita più ``difficile'' ai
brand. In questa maniera i brand hanno bisogno di mantenere un'alta reputazione.

La ``brand reputation'' è la gestione della reputazione di un marchio, che
comprende anche il suo comportamento e la sua apparenza nei social.

\subsection{Gamification}

Con Gamification si intende l'utilizzo di tecniche, regole e dinamiche tipiche
del mondo dei giochi in contesti diversi da quelli ludici. Significa utilizzare
meccaniche e dinamiche di gioco come punti, livelli, reward, missioni e status
all'interno di contesti non gaming per creare engagement e risolvere problemi.
Uno strumento in grado di agire visceralmente \todo{Ha cambiato slide :(}

Esistono diversi universi:
\begin{itemize}
 \item dei giochi
 \item del divertimento (per esempio tra amici)
 \item i social game
 \item lavoro
 \item vita
\end{itemize}

La gamification significa unire il gioco e divertimento con il resto degli
universi. Attenzione! Non bisogna confondere Gamification con Game Design. La
gamification infatti aggiunge meccaniche gaming su un prodotto pensato per
finalità differenti, mentre il game design si occupa di dar vita ad un gioco
vero e proprio, un prodotto profondo ed immersivo che non risulti ne troppo
difficile e neanche troppo complicato da imparare per esempio.

In cosa può essere usata la gamification? Questa si differenzia dalla
pubblicità (pubblicizzo un mio prodotto) e dalla promozione (ad es. prendi 3
paghi 2, sorpresa negli oggetti che si comprano). Nella pubblicità, infatti, si
incentiva la brand awareness, ha bisogno di grossi investimenti e si hanno
guadagni a lungo termine. Le promozioni, invece, non aumentano la brand
awareness, ed incentivano immediatamente le vendite finchè la promozione stessa
dura. La gamification \textbf{non} è una promozione, e coinvolge un sistema a
punteggio/livelli, a cui sono collegate solitamente delle reward e ha un certo
grado di competizione.
La gamification è efficace perché è una tecnica forte per scardinare le
abitudini. Le abitudini sono dei comportamenti radicati nell'individuo che è
propenso a farli (ad esempio cambiare un'abitudine alimentare è faticoso).
Questo funziona bene soprattuto nell'ambito delle diete e del fitness.
Il punto è che rompere o cambiare un'abitudine è molto faticoso, e la
gamification controbilancia elementi altrimenti pensanti e noiosi, mettendo dei
pesi di solito di carattere sociale (ad es. gratificazione personale).

È importante nella vendita di prodotti quindi:
\begin{itemize}
 \item alcuni prodotti vengono scelti e valutati dalla nostra mente razionale
(ad esempio le tariffe telefoniche)
 \item altri parlano con le nostre sensazioni e/o ricordi (trailer del cinema,
cibo), quindi al mid brain, con l'obiettivo di ``parlare'' con le nostre
emozioni.
\end{itemize}

Nota: è importante non mescolare i diversi stili di comunicazione, in quanto
prevale sempre la parte emozionale su quella razionale, in quanto gli aspetti
del mid brain sono più legati alla sopravvivenza.

Per riassumere, la gamification inserisce degli elementi di dialogo
con la mente emozionale.
