Oltre a questo, è presente anche un discorso di etica e morale da tenere in
considerazione.
Le aziende per motivare i manager a fare meglio, hanno un pool di quote stock
che gli vengono dato come ``premio'' per spronarlo a fare sempre meglio. Questo
comportamento si riflette anche con le aziende spin-off, dove una fetta
dell'azienda potrebbe esssere ceduta ad un partecipante esterno, mantenendone
comunque una larga percentuale.

\subsection{Analisi di mercato}

Prima di lanciare una start-up è necessario effettuare un'analisi di mercato,
per vedere se quello che si va a proporre esiste già e con che modalità, se
esistono dei clienti interessati e se ha senso in generale investire in
quell'avventura.

Un modo fallimentare per pensare una start-up è creare un prodotto partendo
dalla mancanza di un elemento già presente nel mercato: è da tenere in
considerazione il fatto che il pubblico potrebbe non essere interessato.

\subsection{Ciclo di vita di una start-up}

\paragraph*{Differenza tra finanziamento e investimento} Nel finanziamento si
ha una somma di capitali spendibili (investibili).

Nei \textbf{finanziamenti} si ottengono dei capitali che o arrivano da
finanziamenti pubblici (bandi, come per esempio le Industrie 4.0) dove i governi
incentivano le aziende a modernizzarsi, e quindi prima le aziende pagano e
investono e poi ricevono dei soldi\footnote{In questo caso si parlano di
finanziamenti a
fondo perduto.} o sono privati. I \textbf{finanziamenti devono essere
solitamente restituiti}, come ad esempio un prestito, dov'è applicato un certo
tasso d'interesse e dov'è possibile restituire pian piano parte del
finanziamento.
Negli \textbf{investimenti}, invece, i capitali non devono essere restituiti,
ma vengono usati per comprare qualcosa (per esempio azioni o quote societarie),
parlando quindi di acquisto di un qualcosa.\\[0.3cm]


\noindent Una delle caratteristiche tipiche delle start-up è che non seguono un
percorso di vita simile a quelle delle aziende comuni, dove da un investimento
di risorse (fisiche, umane, monetarie) si ha la produzione di un prodotto che è
``lineare''.
Nelle start-up invece, avendo caratteristiche diverse (tra cui grandi rischi,
investimenti in nuovi mercati, mancanza di risorse economice necessarie per
far partire l'attività e ricerca continua di investimenti) si caratterizzano
nei seguenti passi:
\begin{enumerate}
 \item business idea (pre-start-up): ci sono due cose importanti in questa fase:
 \begin{itemize}
  \item costituire il team\footnote{Cruciale, la qualità del team è tutto
quello che ``dice'' se c'è la possibilità di realizzare l'idea o meno. In
questo senso, è fondamentale avere una squadra completa prima di iniziare il
progetto, e costituisce la fase centrale.}.
  \item studiare l'idea, il mercato, come si muovono le altre start-up. È anche
importante avere una buona conoscenza del settore in cui si vuole entrare e
anche gli investitori che vogliono partecipare all'avventura. L'elemento chiave
è il \textbf{business model}, trovare ovvero il prodotto che si fa e che valore
esso ha, a chi lo si offre e come si generano flussi di capitali positivi.
 \end{itemize}
 \item bootstrap, ovvero l'avvio, è il momento in cui l'idea diventa qualcosa,
nasce cioè la start-up, che fino a prima era un'idea su carta. L'obiettivo che
si deve raggiungere è il \textbf{MVP}, ovvero il \textit{Minimum viable
product}, cioè un prodotto non completo con delle funzionalità minime (ma
uscito dallo stato prototipale, in grado di sopravvivere nel mercato). Si
differenzia dal prototipo in quanto esso è il primo prodotto completo. L'MVP
presenta caratteristiche minime in quanto permette la produzione in tempi
ristretti: un errore di molta start-up è di uscire con un prodotto completo,
che potrebbe richiedere troppo tempo per essere fatto.

Lo scopo di un MVP è vedere se ci sono dei clienti interessati, avere quindi un
feedback dal mercato, in quanto approcciandosi ad aree nuove di mercato è
importante riuscire ad ottenere delle giuste direzione di mercato. Una volta
recuperato il \textit{feedback di mercato}, se si capisce che la direzione che
si sta prendendo è sbagliata è necessario fare del \textbf{pivoting}, ovvero
cambiare completamente direzione da quella in cui ci si stava muovendo. Nel
mondo delle start-up è normale che ci siano dei continui cambi di direzione, in
quanto si adattano ai feedback di mercato che arrivano: centrare le necessità
del business model al primo colpo è praticamente impossibile.

I \textbf{co-working} sono dei luoghi, strutture e ambienti che mettono a
disposizione degli spazi ad altre aziende/start-up in cambio di qualcosa. Questi
spazi sono importanti perché permettono ai fondatori di start-up di avere degli
ambienti di lavoro in cui è possibile sviluppare e lavorare, molto spesso in
contatto con altre start-up dove idee potrebbero venir scambiate. Ciò che viene
richiesto come pagamento spesso può essere in moneta o in \textit{equity},
ovvero cedendo parte della start-up (di solito fino al 3\%).

Gli \textbf{incubatori} sono la versione evoluta dei co-working, a cui vengono
aggiunti e unificati nuovi servizi (segreteria, telefono, amministrazione,
avvocati, attività di marketing). Gli incubatori evitano di far disperdere
energie alle start-up per permettere di concentrarsi meglio su ciò che
producono. Gli \textbf{acceleratori}, rispetto all'incubatore offrono più
servizi che infrastruttura: anche esse sono organizzazioni (ovviamente) a fini
di lucro, e aiutano le start-up accelerando le loro attività di sviluppo e
business, permettendole di avere accesso ad \textbf{relazioni}, che sono
fondamentali: permettono di ottenere contatti con persone che altrimenti non
sarebbe possibile, e di far decollare il progetto stesso. Solitamente, per
accedere a questi eventi è anche necessario cedere parte della start-up stessa,
solitamente dal 3\% al 4\%. In Italia, se si viene scelti da degli incubatori,
si ricevono dei piccoli \textbf{seed}, che sono del denaro che vengono ceduti
per il progetto che si sta sviluppando\footnote{Si parla solitamente attorno ai
10.000-30.000\euro{}, mentre in america si parla di 300.000-700.000\euro{}. La
differenza, oltre d'investimento, è che gli incubatori in Italia richiedono una
\textit{fee} mensile per il pagamento dei servizi, oltre ad una quota che si
aggira per il 30-35\% della start-up. Gli americani, invece, non richiedono
costi iniziali, ma cercano di guadagnare al momento della exit, ottenendo una
precedenza sulla vendita delle quote societarie. Il fatto è che l'incubatore
Italiano guadagna più tempo tu rimani al suo interno, quello americano il
contrario: il guadagno avviene prima si esce. Nel primo caso non c'è coerenza
negli obiettivi, nel secondo si.}. Incubatori con qualità più alta hanno
selezioni d'accesso più dure, ma presentano una qualità delle start-up
incubate. Questo atteggiamento indica anche una maggiore attenzione al fatto
che le start-up che entrano siano di successo, ovvero si cerca di ottenere un
\textbf{track record}\footnote{Storico delle \textbf{exit} di successo delle
start-up di un incubatore} alto.

Gli \textbf{FFF}\footnote{Ovvero i \textit{friendly, friends and fools},
indicando rispettivamente i parenti/persone strette, gli amici e gli ``stolti''
ovvero chi si fa convincere d'investire subito nell'azienda.} possono aiutare
nella fase di bootstrap.\todo{Non ha detto nient'altro a riguardo!}

 \item a questo punto è possibile cominciare ad andare da investori seri per
poter finire sul mercato. Si parte con attività di visibilità e marketing per
mettere in risalto ciò che viene venduto. I \textbf{Business Angels} possono
aiutare: essi sono un primo gruppo di investori, a cui poi seguono, nella fase
successiva, i Venture Capital. Per le differenze tra i due
vedesi~\ref{startup:crearestartup:bavsvc}. Quando entrano in gioco i business
angel si dice che le quote dei founders si \textit{diluiscono}: il valore della
start-up aumenta, anche se le quote di partecipazione dei fondatori
diminuiscono. Questo, verso la fine, potrebbe portare al termine delle quote di
maggioranza dal punto di vista di founders, anche se possono continuare a
essere considerate azionisti di maggioranza.

 \item la \textbf{traction} è il momento in cui il prodotto viene adottato in
maniera efficace, massiva e ripetitiva da parte del mercato. Quando questa fase
viene raggiunta (fase \textbf{early}) si ha la dimostrazione che la start-up
sta cominciando a funzionare e a fatturare, evolvendosi verso lo stadio
dell'azienda, cominciando a vivere dei prodotti che vende.

A questo punto si punta verso la \textbf{scalabilità}, ovvero
l'internazionalizzazione del prodotto. In questo periodo vengono anche raccolte
metriche per studiare l'andamento delle vendite.

 \item L'ultimo passaggio, definito anche come \textbf{exit}, può essere o
verso altri venture capital o verso altre aziende-colossi o verso lo
sbarco in borsa. I vari investitori trovati durante il percorso possono avere
venduto le quote in qualsiasi delle fasi precedenti.

\end{enumerate}

Ci sono sempre dei casi particolari: non tutte le start-up eseguono i passi qui
di sopra descritti.

\subparagraph*{Differenza tra Business Angels e Venture Capital}
\label{startup:crearestartup:bavsvc}
Business Angels: singolo cittadino, imprenditore o ex-imprenditore (ma anche ex
manager) che possiene un certo capitale. Esso non ha voglia di partire con un
suo progetto imprenditoriale ma è disposto a investire sui progetti di altri.
Oltre ai soldi, essi possono fornire esperienza (anche detto come \textit{smart
money}) e possono fornire consigli, soprattutto se lavorano/lavoravano nello
stesso settore dove la società sta operando.

I venture capital, d'altrocanto, sono delle strutture organizzate
d'investimento: non sono dei privati, ma sono delle banche o dei fondi che di
lavoro investono su progetti di rischio. Le grandi differenze tra queste due
entità sono:
\begin{itemize}
 \item i Business Angles tendono a essere coinvolti molto di più nella dinamica
della start-up: esso vuole essere dentro il progetto, ma può essere anche un
problema se la persona è molto ``ingombrante''. È importante quindi fare molta
attenzione su chi vuole investire
 \item i Venture Capital vogliono più il rendiconto e il guadagno finale, e
sono quindi molto più legati all'aspetto finanziario.
 \item i Venture Capital dispongono di più finanziamenti rispetto a un
Business Angels un business angel in Italia investe dai 10.000 ai
50.000\euro{}, mentre un venture capital parte almeno da 2 milioni di \euro{}
d'investimento.
\end{itemize}

È importante notare come le VC (venture capital) possano porre delle clausole
di investimento molto restrittive. \textbf{Advisors} possono fornire ulteriori
garanzie sul team di sviluppo iniziale. Un advisor è colui che, formalmente, è
in grado di fornire assicurazioni da una parte, e consigli per il team di
sviluppo dall'altra.
