

Oltre a questo, è presente anche un discorso di etica e morale da tenere in
considerazione.

Le aziende solitamente riservano una certa quantità di \textbf{quote stock}
tenute per essere cedute a esterni o interni meritevoli per aumentare la loro
partecipazione nel progetto o come ``premio'' per spronarlo a fare sempre
meglio. Questo comportamento è tipico anche delle spin-off dove una fetta
dell'azienda potrebbe essere ceduta ad un partecipante esterno, mantenendone
comunque una larga percentuale.

Negli spin-off universitari invece solitamente ci sono problemi, poiché gli
ideatori che diventano successivamente soci, non si affidano ad un manager pur
non essendo del settore.

\subsection{Analisi di mercato}

Prima di lanciare una start-up è necessario effettuare un'analisi di mercato,
per vedere se quello che si va a proporre esiste già e con che modalità, se
esistono dei clienti interessati e se ha senso in generale investire in
quell'avventura.

Un modo fallimentare per pensare una start-up è creare un prodotto partendo
dalla mancanza di un elemento già presente nel mercato: è da tenere in
considerazione il fatto che il pubblico potrebbe non essere interessato.

Le idee delle startup sono spesso caratterizzate avere una grossa opportunità,
poiché è un mercato ancora inesplorato, ma per questo anche un grandissimo
rischio, poiché non ci sono abbastanza informazioni per effettuare un'analisi
di mercato completa. Spesso le startup non hanno le risorse iniziali per
partire, oppure ce la potrebbero fare ma ciò richiederebbe molto tempo. Le
startup sono spesso associate alla tecnologia, poiché è un settore dove ci sono
cambiamenti molto veloci.

\subsection{Ciclo di vita di una start-up}

\paragraph*{Differenza tra finanziamento e investimento} 

Nei \textbf{finanziamenti} si ottengono dei capitali che possono arrivare o da
finanziamenti/bandi pubblici (come per esempio le Industrie 4.0) dove i governi
incentivano le aziende a modernizzarsi, e quindi prima le aziende pagano e
investono e poi ricevono dei soldi indietro\footnote{In questo caso si parla
di finanziamenti a fondo perduto.} o sono privati. I \textbf{finanziamenti
devono essere solitamente restituiti}, come ad esempio un prestito, ai quali è
applicato un certo tasso d'interesse che varia, solitamente, a seconda del
tempo di restituzione del finanziamento. I soldi in più che vengono restituiti
sono il guadagno di chi ha fornito il finanziamento.

Negli \textbf{investimenti}, invece, i capitali \textbf{non devono essere
restituiti}, ma vengono usati per comprare qualcosa (per esempio azioni o quote
societarie), parlando quindi di acquisto di un qualcosa. In questo caso il
guadagno dell'investitore è di due tipi:
\begin{itemize}
	\item dividendi (poco), e
	\item eventuale \textit{exit} dall'azienda (spesso molto più redditizio)
\end{itemize}

\noindent Una delle caratteristiche tipiche delle start-up è che non seguono un
percorso di vita simile a quelle delle aziende comuni, nelle quali si ha un
investimento di risorse (fisiche, umane, monetarie) per la produzione di un
prodotto, che viene venduto per acquistare altre risorse, quindi una crescita
``lineare''.

Nelle start-up invece, si hanno caratteristiche diverse (tra cui grandi rischi,
investimenti in nuovi mercati, mancanza di risorse economiche necessarie per
far partire l'attività e ricerca continua di investimenti) e si cercano
differenti round di investimenti. I passi di crescita delle startup sono i
seguenti (ognuno di questi possono essere ripetuti):
\begin{enumerate}
 \item \textbf{business idea (pre-start-up)}: in questo passo è necessario
 definire per bene l'idea di prodotto e il modo per guadagnare da quest'ultimo.
 Ci sono due cose importanti in questa fase:
 \begin{itemize}
  \item costituire il team. Nella prima parte di costituzione dell'azienda
  infatti gli investitori non decidono principalmente se investire o meno su di
  un progetto in base all'idea ma in base al team. La qualità del team
  determina se c'è la possibilità di realizzare l'idea o meno. Per questo è
  fondamentale presentarsi agli investitori solo con una squadra completa.

  \item studiare l'idea, la sua fattibilità, quanto può valere, il mercato,
  come si muovono le altre start-up. È anche importante avere una buona
  conoscenza del settore in cui si vuole entrare e intervistare esperti,
  investitori e chi conosce bene un certo mercato. 
 \end{itemize}
 L'obiettivo di questa fase è definire il \textbf{business model}, trovare
 ovvero il prodotto che si vuole sviluppare e che valore ha, cosa mi serve per
 produrlo, a chi lo si offre e come si generano flussi di capitali positivi.
 
 \item bootstrap, ovvero l'avvio, è il momento in cui l'idea diventa qualcosa,
 nasce cioè la start-up. Fin prima, infatti, l'idea era solo su carta.
 L'obiettivo è la creazione del \textbf{MVP}, ovvero il \textit{Minimum Viable
 Product}, cioè un prodotto non completo con delle funzionalità chiave minime (
 non è un prototipo, in quanto un prototipo è il primo prodotto completo).
 L'MVP presenta caratteristiche minime in quanto permette la sua produzione in
 tempi ristretti: un errore di molte start-up è voler uscire con un prodotto
 troppo rifinito per essere un MVP, e ciò richiede troppo tempo per essere
 fatto.
 
 Lo scopo di un MVP è di ottenere sia un feedback dal mercato sia
 un avvicinamento di eventuali clienti interessati. Ciò è fondamentale in
 quanto approcciandosi ad aree nuove di mercato è importante ottenere delle
 giuste direzione dallo stesso. Nel caso in cui il mercato indichi che la
 direzione di mercato scelta è errata è necessario fare \textbf{pivoting},
 ovvero cambiare la direzione in cui ci si stava muovendo. Nel mondo delle
 start-up è normale che ci siano dei continui cambi di direzione: centrare le
 necessità del business model al primo colpo è praticamente impossibile. Spesso
 se una startup non fa pivoting vuol dire che da poca importanza ai feedback
 del mercato oppure non ne vengono proprio raccolti. Un tipo particolare di
 pivoting è lo zoom-in: di un prodotto con tante funzionalità si individuano
 quelle interessanti e le altre si abbandonano.

 Ci sono vari aiuti ad una startup per il bootstrap.

 I \textbf{co-working} sono dei luoghi, strutture e ambienti che messi a
 disposizione ad aziende/start-up in cambio di qualcosa.
 Questi spazi sono importanti perché permettono ai fondatori di start-up di
 avere degli ambienti di lavoro in cui è possibile sviluppare e lavorare, molto
 spesso entrare in contatto con altre start-up dove idee potrebbero venire
 fuori. Ciò che viene richiesto come pagamento spesso può essere in moneta o in
 \textit{equity}, ovvero cedendo parte della start-up (di solito fino al 3\%).
 Ciò sono alcune eccezioni di co-working statali gratuiti. I vantaggi di tali
 luoghi è soprattutto la possibilità di comunicazione tra le startup: le
 aziende alla fin fine hanno sempre gli stessi problemi e è utile sapere come
 altre persone li hanno superati.

 Gli \textbf{incubatori} (come ad esempio H-Farm) sono la versione evoluta dei
 co-working, a cui vengono aggiunti e unificati certi servizi (segreteria,
 telefono, amministrazione, avvocati, attività di marketing). Gli incubatori
 evitano di far disperdere energie alle start-up (ad esempio uno startupper
 potrebbe non sapere come funziona la fatturazione e quindi gli può venire
 offerto qualcuno che se ne occupa) per permettere di concentrarsi meglio su
 ciò che producono. Struttura e servizi vengono offerti dietro compenso o quote
 societarie.

 Gli \textbf{acceleratori}, rispetto all'incubatore offrono più servizi ma non
 l'infrastruttura: anche esse sono organizzazioni (ovviamente) a fini di lucro,
 e aiutano le start-up accelerando le loro attività di sviluppo e business,
 permettendole di avere accesso a \textbf{relazioni}, che sono fondamentali:
 permettono di ottenere contatti con persone che altrimenti non sarebbe
 possibile, e di far decollare il progetto stesso. Uno degli obiettivi di
 questo passo evolutivo della startup è creare un network di persone ``utili''
 a far crescere l'azienda. Solitamente, per accedere a questi servizi è anche
 necessario cedere parte delle quote della startup, solitamente dal 3\% al 4\%.

 In Italia, se si viene scelti da degli incubatori, si ricevono dei piccoli
 \textbf{seed}, che sono del denaro che viene investito nel progetto che si sta
 sviluppando. Si parla solitamente attorno ai 10.000-30.000\euro{}, molti meno
 dei 300.000-700.000\euro{} degli USA. La differenza, oltre alla quota di
 investimento, è che gli incubatori in Italia richiedono sia una \textit{fee}
 mensile per il pagamento dei servizi, che quota molto alta della startup, che
 si aggira intorno al 30-35\%. Gli americani, invece, non richiedono costi
 iniziali, ma cercano di guadagnare al momento della exit, ottenendo una
 precedenza sulla vendita delle quote societarie. Inoltre la quota richiesta è
 nettamente più bassa, circa del 5-10\%. In Italia quindi l'incubatore guadagna
 tanto più quanto più tempo rimani al suo interno, nel modello americano il
 guadagno si ha al momento dell'exit. Nel primo caso, quindi, non c'è coerenza
 negli obiettivi, nel secondo si.

 Incubatori e acceleratori con qualità più alta hanno selezioni d'accesso più
 dure, ma ciò si riflette in una maggiore qualità delle start-up incubate.
 Questo atteggiamento indica anche una maggiore attenzione al fatto che le
 start-up che entrano siano di successo, ovvero si cerca di ottenere un
 \textbf{track record}\footnote{Storico delle \textbf{exit} di successo delle
 start-up di un incubatore} alto.

 Gli \textbf{FFF}\footnote{Ovvero \textit{family, friends and fools},
 indicando rispettivamente i parenti/persone strette, gli amici e gli ``stolti''
 ovvero chi si fa convincere d'investire subito nell'azienda.} possono aiutare
 nella fase di bootstrap, dando un piccolo apporto economico iniziale.

 \item \textbf{seed}: implementato l'MVP e fatti eventuali pivoting, è
 possibile cominciare ad andare da investitori seri per poter finire sul mercato.
 Si parte con attività di visibilità e marketing per mettere in risalto ciò che
 viene venduto. In questa fase sono importanti i \textbf{Business Angels}: essi
 sono un primo gruppo di investitori, a cui poi seguono, nella fase successiva, i
 Venture Capital\footnote{Per le differenze tra i due
 vedesi~\ref{startup:crearestartup:bavsvc}}. Oltre il loro denaro è
 importante la loro esperienza e, se lavoravano in un settore troppo differente
 da quello della startup, il loro network di conoscenze. Quando entrano in
 gioco i business angel si dice che le quote dei founders si
 \textit{diluiscono}: il valore della start-up aumenta, anche se le quote di
 partecipazione dei fondatori diminuiscono. Questo, verso la fine, potrebbe
 portare al termine delle quote di maggioranza dal punto di vista di founders,
 anche se possono continuare a essere considerati azionisti di maggioranza.

 \item la \textbf{traction} è il momento in cui il prodotto viene adottato in
 maniera efficace, massiva e ripetitiva da parte del mercato, e possibilmente
 c'è una costante crescita nelle vendite. Quando questa fase viene raggiunta 
 (fase \textbf{early}) si ha la dimostrazione che la start-up sta cominciando a
 funzionare e a fatturare, evolvendosi verso lo stadio dell'azienda,
 cominciando a vivere dei prodotti che vende.

 A questo punto si punta verso la \textbf{scalabilità}, ovvero
 l'internazionalizzazione del prodotto. In questo periodo vengono anche raccolte
 metriche per studiare l'andamento delle vendite.

 \item L'ultimo passaggio, definito anche come \textbf{exit}, può essere o
 verso altri venture capital o verso altre aziende-colossi o verso lo sbarco in
 borsa. I vari investitori trovati durante il percorso possono avere
 venduto le quote in qualsiasi delle fasi precedenti.

\end{enumerate}

Ci sono sempre dei casi particolari: non tutte le start-up eseguono i passi qui
di sopra descritti.

\subparagraph*{Differenza tra Business Angels e Venture Capital}
\label{startup:crearestartup:bavsvc}
Business Angels: singolo cittadino, imprenditore o ex-imprenditore (ma anche ex
manager) che possiede un certo capitale. Esso non ha voglia di partire con un
suo progetto imprenditoriale ma è disposto a investire sui progetti di altri.
Oltre ai soldi, essi possono fornire esperienza (anche detto come \textit{smart
money}) e possono fornire consigli, soprattutto se lavorano/lavoravano nello
stesso settore dove la società sta operando.

I venture capital, d'altro canto, sono delle strutture organizzate
d'investimento: non sono dei privati, ma sono delle banche o dei fondi che di
lavoro investono su progetti di rischio. Le grandi differenze tra queste due
entità sono:
\begin{itemize}
 \item i Business Angles tendono a essere coinvolti molto di più nella dinamica
della start-up: esso vuole essere dentro il progetto, ma può essere anche un
problema se la persona è molto ``ingombrante''. È importante quindi fare molta
attenzione su chi vuole investire;
 \item i Venture Capital vogliono più il rendiconto e il guadagno finale, e
 sono quindi molto più legati all'aspetto finanziario: i business angel possono
 quindi essere più comprensivi rispetto a dei fallimenti negli obiettivi
 aziendali.
 \item i Venture Capital dispongono di più finanziamenti rispetto a un
 Business Angels. Quest'ultimi in Italia investono dai 10.000 ai
 50.000\euro{}, mentre un venture capital parte almeno da 2 milioni di \euro{}
 d'investimento.
\end{itemize}

È importante notare come le VC (venture capital) possano porre delle clausole
di investimento molto restrittive. Un esempio è la clausola
\textit{bad leaver}: in seguito a degli eventi negativi (cause penali, azioni
strettamente contrarie agli interessi degli azionisti, mancato raggiungimento
degli obiettivi minimi) gli investitori con \textit{preferred stock} possono
prendere tutte le quote dei fondatori gratuitamente. Grazie le
\textit{preferred stock} gli investitori, in caso di un'offerta davvero
vantaggiosa possono decidere di vendere le quote anche dei fondatori che
hanno \textit{common stock}.

\textbf{Advisors} possono fornire ulteriori garanzie sul team di sviluppo
iniziale. Un advisor è colui che, formalmente, è in grado di fornire
assicurazioni da una parte, e consigli per il team di sviluppo dall'altra.
