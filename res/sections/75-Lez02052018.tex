\section{Comunicazione}
In colloqui lavorativi esistono anche situazioni in cui l'intervistato viene
posto sotto pressione appositamente per vedere le sue reazioni e come sarà in
grado di sopportare lo stress. Di queste non se ne vedono molte in giro in
quanto sono difficili da portare avanti, e possono ottenere il risultato
opposto, ovvero che i candidati ``scappano'' non presentandosi.

\subsection{Non verbale}

Per comunicazione non verbale s'intende l'intonazione della voce e il modo di
porsi con le persone. La parola, infatti, conta per il 20\% in un colloquio,
mentre il modo in cui si parla è importante, in quanto è in grado di
trasmettere emotività.

La comunicazione non verbale è gestibile in parte, ma non è interamente
controllabile, questo anche perché deriva dai nostri progenitori, ed è insita
nel nostro modo di comunicare.

Studiare la comunicazione non verbale è una disciplina estremamente efficace,
ma poco scientifica al giorno d'oggi. La PNL (programmazione neurolinguistica)
viene usata soprattutto dai venditori per migliorare l'efficacia della vendita,
e si occupa dell'aspetto e del portamento che si ha.

\paragraph*{Perché leggere la comunicazione non verbale è così importante?} Le
menzogne sono insite nell'essere umano, che riesce a mentire meglio tramite
l'utilizzo del viso (che è la parte dopotutto più vista dalle
persone).\\[0.3cm]

\noindent Le persone perdono il controllo della comunicazione non verbale
distanziandosi sempre di più dalla bocca: i piedi infatti sono le zone più
difficilmente controllabili durante una menzogna. D'altro canto, la
comunicazione non verbale non è detta che sia sempre correlata ad un significato
in particolare. Difatti, ogni persona ha una propria \textit{baseline}
posturale: bisogna prestare attenzione al cambiamento di posizione, e bisogna
verificare che diversi segnali portino allo stesso significato.

\begin{example}[Esperimento della matita]
È stato condotto tempo fa un esperimento in cui due gruppi vedeva una sitcom
in cui una parte aveva una matita in bocca e l'altra no. Si è scoperto che il
gruppo con la matita in bocca trovava la sitcom più divertente.

\noindent Questo perché la matita sforzava i muscoli della bocca spingendo il
gruppo a ridere di più.
\end{example}

\todo{Aggiungere parte del mirroring che mi sono perso} Fare mirroring è
sconsigliato, in quanto implica un livello di sicurezza abbastanza alto. Delle
piccole tecniche di mirroring possono essere applicati, senza essere eccessivi,
sul vestiario: adattare il modo di vestire in maniera coerente con
l'interlocutore è una piccola tecnica di mirroring.
