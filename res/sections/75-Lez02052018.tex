\section{Comunicazione}
I colloqui lavorativi sono spesso situazioni stressanti ma ci sono anche delle
particolari interviste, chiamate \emph{stress interview}, nelle quali
l'intervistato viene posto sotto pressione appositamente per vedere le sue
reazioni e come sarà in grado di sopportare lo stress. Per fare ciò
l'intervistatore fa domande scomode e ha reazioni appositamente negative. Sono
molto rare sia perché sono difficili da gestire anche per l'intervistatore, sia
perché possono ottenere il risultato opposto, ovvero che i candidati
``scappino'' poiché il messaggio che si ha passato è di un'azienda che si è
comportata male.

Di un colloquio è inoltre importante ricordare che molte informazioni, da parte
di un intervistatore attento, possono essere raccolte dopo l'intervista, al
momento in cui viene bevuto un caffè o in un momento in cui si abbassa la
tensione, tramite la comunicazione non verbale nel momento in cui ci
rilassiamo.

\subsection{Non verbale}

Per comunicazione non verbale s'intende l'intonazione della voce e il modo di
porsi con le persone. La parola, infatti, conta per il 20\% in un colloquio,
mentre il resto è trasmesso dall'intonazione (prosodia), gesti, come siamo
vestiti e la nostra postura. Questo ci fa capire come una comunicazione per
email faccia perdere molto dell'emozione di quello che vogliamo dire. Le chat
permettono di recuperarne una parte come emoji e animazioni.

La comunicazione non verbale è gestibile in parte, ma non è interamente
controllabile, questo anche perché deriva dai nostri progenitori, e fa parte
del nostro modo di relazionarsi.

Studiare la comunicazione non verbale è una disciplina estremamente efficace,
ma poco scientifica al giorno d'oggi. La PNL (programmazione neurolinguistica)
viene usata soprattutto dai venditori per migliorare l'efficacia della vendita,
e si occupa dell'aspetto e del portamento che si ha.

\paragraph*{Perché leggere la comunicazione non verbale è così importante?} Le
menzogne sono insite nell'essere umano e ci viene insegnato fin da piccoli a
raccontarne, e la parte del corpo che più ci aiuta è il viso.\\[0.3cm]

\noindent Un errore tipico è dire che un singolo gesto o posizione può dirci
quello che una persona vuole trasmettere: ci sono tanti fattori, quali clima,
abitudini, comodità, \dots{} Quindi se vogliamo capire cosa una persona vuole
comunicarci con il suo atteggiamento ci sono 3 passi:
\begin{enumerate}
\item capire la postura tipica di una persona (baseline);
\item capire se/quando ci sono cambi nella postura;
\item scorgere più gesti che indicano la medesima cosa.
\end{enumerate}
Le persone perdono il controllo della comunicazione non verbale
distanziandosi sempre di più dalla bocca: i piedi infatti sono le zone più
difficilmente controllabili durante una menzogna o in caso in cui ci sentiamo a
disagio.

Facepalm, stropicciarsi/toccarsi gli occhi o i capelli, giocare con la collana,
piedi che puntano verso l'uscita più vicina sono tutti segnali di
disapprovazione/disagio/protezione. Al contrario piedi incrociati, se si
sollevano leggermente le punte, mani dietro la schiena invece trasmettono
apertura/felicità/che ci si sente a proprio agio.

\begin{example}[Esperimento della matita]
È stato condotto tempo fa un esperimento in cui due gruppi vedeva una sitcom
in cui una parte aveva una matita in bocca e l'altra no. Si è scoperto che il
gruppo con la matita in bocca trovava la sitcom più divertente.

\noindent Questo perché la matita sforzava i muscoli della bocca spingendo il
gruppo a ridere di più.
\end{example}

\paragraph*{Mirroring e neuroni specchio}
Gli uomini sono animali sociali e hanno sviluppato alcuni neuroni particolari,
detti \emph{neuroni specchio}, che ci portano a imitare i nostri simili.
Servono principalmente a due cose: addestrare i bambini e imparare a gestire le
relazioni sociali. Questo viene sfruttato dalla tecnica del \emph{mirroring},
che prevede di emulare la gestualità e il linguaggio del nostro interlocutore
per metterlo a suo agio e possibilmente trasmettere il proprio messaggio in
maniera più efficace. Questa tecnica è molto difficile, in quanto implica un
livello di sicurezza abbastanza alto. Delle piccole tecniche di mirroring
possono essere applicate, senza essere eccessivi, sul vestiario: adattare il
modo di vestire in maniera coerente con l'interlocutore ne è un piccolo esempio.
